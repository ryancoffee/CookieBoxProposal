Continued and significant progress in the generation and use of novel ultrashort X-ray pulses at the LCLS X-ray Free Electron Laser (XFEL) ensures that LCLS-II will maintain its international leadership. 
We are sure to continually develop even more novel modes based on the new variable gap undulators, split and Delta undulator configurations that allow multiple polarizations, ``fresh slice'' lasing for dual pulses, and even attosecond pulse generation. 
Multi-color, multi-polarization and multiple x-ray pulses with exquisite temporal control are all featured priority areas for LCLS-II which cites as a principle need being attosecond time resolved photo and Auger electron emission in order to directly capture detailed dynamics of correlated electron motion. 
We will no longer be making molecular movies, but rather movies of the electrons moving around and through those molecules.
To do this, however, one requires a diagnostic that can recover the exact temporal profile for each Self-Amplification of Spontaneous Emission (SASE) pulse. 
Such a single shot diagnostic must recover potentially complicated pulse shapes for on-the-fly data sorting and veto.

So called ``angular streaking'' was recently identified as the likely method to deliver the needed pulse characterization.
In actuality, it provides the foundation for a comprehensive attosecond experimental paradigm. 
Familiar in the high-harmonic generation laser community, angular streaking uses a long-wavelength streaking laser to provide a ``clock'' against which attosecond electron dynamics can be measured. 
Given the similar requirements, we propose to address the pulse diagnostic needs as well as provide this basis for attosecond resolved electron spectroscopy at LCLS-II.

We propose a new generation of attosecond diagnostic capability, one that is tailored to the unique XFEL pulses existing and expected, having learned from our recently demonstrated reconstruction of attosecond scale pulse structures of LCLS. 
We found that the synchrotron-optimized detector array that was used for that initial demonstration suffers multiple shortcomings for FEL use; limitations that have proven the most challenging impediments to accurate pulse reconstruction. 

The scope of this proposed project is therefore the Research and Development required to develop an XFEL optimized angular array of electron Time-of-Flight (eTOF) spectrometers that will meet the stringent needs of LCLS-II. 
The new detector array will be optimized specifically for single-shot angular streaking measurements at the FEL while minimizing the inter-detector cross-talk experienced in the previous design.
We target a spectral resolution of 0.25 eV by improving the sensor electronics and by integrating on-board signal processing that is specifically matched to the LCLS-II data reduction pipeline. 
Furthermore, we will design for a new feature whereby the eTOFs are capable of analyzing multiple spectral windows, each of high energy resolution, to accommodate two-color double pulses from widely detuned variable gap undulators. 
This multiple window feature is a key development for the sake of element-specific tracking of electron transfer and charge migration, unlocking the core element of the LCLS-II attosecond science program. 
Furthermore, we will design to accommodate also the split undulator method in combination with Delta undulator production of variably polarized pulses, ensuring sub-spike polarization analysis.

The output from this project will not only provide the design basis for attosecond resolving single-shot x-ray pulse reconstruction but also an advanced instrumentation concept as a platform for core and future LCLS-II science.

%\paragraph{Cost and Scope:}
%The cost of this two year project is \$307k per year for each of 2 years.
%This accommodates one Research Associate for the duration of the project with the majority of the cost reserved for the materials and supplies used to build eTOFs, their corresponding analog electronics, and the digital signal processing that could integrate with the expected LCLS-II data pipeline.




%Multi-color, multi-polarization and multiple x-ray pulses with temporal control are all featured as LCLS-II needs for the future of Free-Electron Laser (FEL) science. 
%Attosecond resolution for photo and Auger electron time-dependent emission spectra aims to time resolve correlated electron motion, thus forming a core topic for LCLS-II. 
%To fully capitalize on the nascent field of xFEL-based atto-science, one requires a diagnostic that can recover the exact temporal profile for each Self-Amplification of Spontaneous Emission (SASE) pulse.
%Beyond SASE, LCLS continues to develop novel modes of operation based on the new variable gap undulators, or split and Delta undulators, Freshslice lasing, or even xFEL attosecond pulse generation from beam-based coherent radiation. 
%A single shot diagnostic, with attosecond temporal resolution, to recover these complicated pulse shapes for on-the-fly data sorting and veto is therefore imperative.
%Furthermore, the angular streaking method is likely to be the full experimental paradigm, not simply the pulse characterization diagnostic, e.g. the long-wavelength streaking laser provides the ``clock'' against which the attosecond dynamics are measured in the experiment.
%Both cases have similar requirements and therefore we propose to address not only the diagnostic capability, but actually trigger an angular streaking paradigm for electron spectroscopy at LCLS-II.
% 
%We propose a next generation of the recently demonstrated attosecond reconstruction of the single pulse structure of LCLS.
%Based on recent theoretical explorations into FEL pulse reconstruction, we find that the synchrotron-optimized detector array that was used for that initial demonstration experiment suffers multiple shortcomings for FEL use; limitations that have proven the most challenging impediments to x-ray pulse reconstruction.
%The scope of this proposed project is therefore the Research and Development required to produce an angular array of 16 re-designed electron Time-of-Flight (eTOF) spectrometers. 
%The new detector array will be optimized specifically for single-shot FEL measurements, minimizing detector cross-talk, targeting a spectral resolution of 0.25 eV, and improving the sensor electronics with integrated on-board signal processing that is matched to the LCLS-II data reduction pipeline.
%Furthermore, we will design a new feature whereby the eTOFs are capable of analyzing multiple windows, each of high energy resolution, but capable of spanning up to many hundreds of electron volts, thus accommodating two-color double pulses from widely detuned variable gap undulators.
%In order to accommodate also the split undulator method in combination with Delta production of variably polarized pulses, we will ensure that each of the two color pulses can be analyzed as circular or linear polarization.
%
