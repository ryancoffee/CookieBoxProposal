Substantial progress has recently been made in the generation and utilization of ultrashort pulse X-rays from the LCLS X-ray Free Electron Laser (XFEL). 
Multi-color, multi-polarization and multiple x-ray pulses with exquisite temporal control are all featured as priority areas for LCLS-II to maintain our international leadership in this field. 
For example, attosecond resolution for photo and Auger electron emission spectra is needed to capture the detailed dynamics of correlated electron motion. 
To fully capitalize on this nascent field of XFEL-based atto-science, one requires a diagnostic that can recover the exact temporal profile for each Self-Amplification of Spontaneous Emission (SASE) pulse. 
Beyond SASE, LCLS continues to develop novel modes of operation based on the new variable gap undulators, split and Delta undulators to control polarization, freshslice lasing for dual pulses, and even attosecond pulse generation. A single shot diagnostic, with attosecond temporal resolution, to recover these complicated pulse shapes for on-the-fly data sorting and veto is therefore imperative, given the stochastic nature of XFEL pulses.

Work to date has identified the method of ``angular streaking'' as likely to be able to move beyond simple pulse characterization, to provide the basis for a comprehensive experimental paradigm. 
Using this method, a long-wavelength streaking laser can provide a ``clock'' against which attosecond dynamics are measured in the experiment. 
Both cases have similar requirements and therefore we propose to address not only the pulse diagnostic capability, but actually provide the basis for an angular streaking paradigm for electron spectroscopy at LCLS-II.

We propose a next generation of diagnostic capability, learning from the recently demonstrated reconstruction of the single pulse structure of LCLS. 
Based on recent theoretical explorations into FEL pulse reconstruction, we find that the synchrotron-optimized detector array that was used for that initial demonstration experiment suffers multiple shortcomings for FEL use; limitations that have proven the most challenging impediments to x-ray pulse reconstruction. 
The scope of this proposed project is therefore the Research and Development required to produce an angular array of re-designed electron Time-of-Flight (eTOF) spectrometers to meet the stringent needs of LCLS-II. 
The new detector array will be optimized specifically for single-shot FEL measurements, minimizing detector cross-talk, targeting a spectral resolution of 0.25 eV, and improving the sensor electronics with integrated on-board signal processing that is matched to the LCLS-II data reduction pipeline. 
Furthermore, we will design a new feature whereby the eTOFs are capable of analyzing multiple spectral windows, each of high energy resolution, but capable of spanning up to many hundreds of electron volts, thus accommodating two-color double pulses from widely detuned variable gap undulators. 
This is essential for element-specific tracking of charge migration, underpinning a major element of the LCLS-II attoscience program. 
In order to accommodate also the split undulator method in combination with Delta production of variably polarized pulses, we will ensure that each of the two color pulses can be analyzed as circular or linear polarization.

The output from this project will provide the design basis for a new instrument platform for LCLS-II science.




%Multi-color, multi-polarization and multiple x-ray pulses with temporal control are all featured as LCLS-II needs for the future of Free-Electron Laser (FEL) science. 
%Attosecond resolution for photo and Auger electron time-dependent emission spectra aims to time resolve correlated electron motion, thus forming a core topic for LCLS-II. 
%To fully capitalize on the nascent field of xFEL-based atto-science, one requires a diagnostic that can recover the exact temporal profile for each Self-Amplification of Spontaneous Emission (SASE) pulse.
%Beyond SASE, LCLS continues to develop novel modes of operation based on the new variable gap undulators, or split and Delta undulators, Freshslice lasing, or even xFEL attosecond pulse generation from beam-based coherent radiation. 
%A single shot diagnostic, with attosecond temporal resolution, to recover these complicated pulse shapes for on-the-fly data sorting and veto is therefore imperative.
%Furthermore, the angular streaking method is likely to be the full experimental paradigm, not simply the pulse characterization diagnostic, e.g. the long-wavelength streaking laser provides the ``clock'' against which the attosecond dynamics are measured in the experiment.
%Both cases have similar requirements and therefore we propose to address not only the diagnostic capability, but actually trigger an angular streaking paradigm for electron spectroscopy at LCLS-II.
% 
%We propose a next generation of the recently demonstrated attosecond reconstruction of the single pulse structure of LCLS.
%Based on recent theoretical explorations into FEL pulse reconstruction, we find that the synchrotron-optimized detector array that was used for that initial demonstration experiment suffers multiple shortcomings for FEL use; limitations that have proven the most challenging impediments to x-ray pulse reconstruction.
%The scope of this proposed project is therefore the Research and Development required to produce an angular array of 16 re-designed electron Time-of-Flight (eTOF) spectrometers. 
%The new detector array will be optimized specifically for single-shot FEL measurements, minimizing detector cross-talk, targeting a spectral resolution of 0.25 eV, and improving the sensor electronics with integrated on-board signal processing that is matched to the LCLS-II data reduction pipeline.
%Furthermore, we will design a new feature whereby the eTOFs are capable of analyzing multiple windows, each of high energy resolution, but capable of spanning up to many hundreds of electron volts, thus accommodating two-color double pulses from widely detuned variable gap undulators.
%In order to accommodate also the split undulator method in combination with Delta production of variably polarized pulses, we will ensure that each of the two color pulses can be analyzed as circular or linear polarization.
%
