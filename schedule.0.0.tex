\subsection*{Organization of Major Activities}\addcontentsline{toc}{subsection}{Organization of Major Activities}

We schedule the development of the attosecond angular streaking effort to coincide with the attosecond x-ray pulse developments at LCLS.
We will benchmark the resolution of the angular streaking method by measuring attosecond and multi-pulse, multi-color, multi-polarization mode of Ref.~\cite{Lutman2016} to benchmark our ability to directly measure the full time-dependent spectrum and polarization of novel shaped x-ray FEL pulses.


\begin{table}
\caption{\label{tab::schedule} Timeline of the R\&D schedule. Project Years (PY) are sub-divided into 6 month blocks. }
\begin{tabular}{l|l|l|l}
\hline
PY0.0 & PY0.5 & PY1.0 & PY1.5\\
\hline
somethign & else& and & this\\
somethign & else& and & this\\
\end{tabular}
\end{table}


\paragraph*{Period 0.0: 7/1/2018 -- 12/31/2018\\}
\textbf{Objective \ref{obj::streaking}: }
Test 2$\mu$m streaking with attosecond pulses from LCLS-I using Axial VMI \cite{Siqi2018}.\\
\textbf{Publication(s): }
``Thomas Paper''

\paragraph*{Period 0.5: 1/1/2019 -- 6/31/2019\\}
\textbf{Objective \ref{obj::streaking}: }
Design new CookieBox detector, use existing data to develop physics based algorithms for streaming analysis.\\
\textbf{Publication(s): }
``Martin and Rupert''
``Gregor Paper''

\paragraph*{Period 1.0: 7/1/2019 -- 12/31/2019\\}
Construct new modular CookieBox detector for LCLS-II.\\
\textbf{Publication(s): }
``Transfer Learning paper''

\paragraph*{Period 1.5: 1/1/2020 -- 6/31/2020, instrument commissioning\\}
Implement on-board FPGA processing and demonstrate/simulate calibration learning, detail implementation plan for LCLS-II.\\
Build passive CEP locked 3 $\mu$m pulses, commission new CookieBox detector for LCLS-II, develop algorithm for streaming analysis, benchmark 150 attosecond resolution.\\
\textbf{Publication(s): }
``Benchmarking/Novel algorithm for real-time x-ray pulse characterization at the LCLS-II.''

\paragraph*{Period 5: 7/15/2021 -- 7/14/2022, LCLS-II operating\\}
Install on LCLS-II with on-board processing, benchmark resolution with NEXAFS of 3 $\mu$m laser dressing of N$_2$O, and develop the implementation matrix for various locations in LCLS-II, and explore high rep-rate application to hard x-ray regime.\\
\textbf{Publication(s): }
``Streaming attosecond resolution x-ray pulse characterization: spectrum, polarization, and time.''
``Shaped attosecond x-ray FEL pulses for nonlinear x-ray science''
``Direct observation of laser-mixing of valence electronic symmetries.''




\subsection*{Responsibilities of key project personnel}\addcontentsline{toc}{subsection}{Responsibilities of key project personnel}

The PI will be responsible for organizing the major efforts including algorithm development, optical design, and coordination of engineering design and hardware construction.
He is expected to contribute 65\% of his effort to this project.

The first RA/Project Scientist will be hired at the 75\% level with 25\% of his or her effort will be shared with closely related Accelerator R\&D projects.
He or she will work primarily on the optical construction and execution of the experiments in the first two years for the interferometric timing effort.
In addition, he or she will organize an initial preliminary benchmark test of attosecond angular streaking based on the Viefhaus CookieBox that will be borrowed from DESY.
He or she will begin the effort of designing the CookieBox detector that will be used in the attosecond angular streaking effort.
A second RA/Project Scientist will take over the CookieBox detector construction and will lead the attosecond streaking beamtime preparations in years 3-5.
He or she will focus also on the optical design and fabrication responsibilities for the final dedicated CookieBox detector and required laser implementation.

We intend to recruit two graduate students from Stanford University to participate in the project.
We will seek funding through student research fellowships and LCLS-sponsored student outreach programs.
Related research by the PI has attracted numerous externally funded post-doctoral fellows and visiting scientists, most notably Wofram Helml (Marie Curie Foundation), Anton Lidahl (Wallenberg Foundation), and Markus Ilchen (Volkswagen Foundation).
We expect these collaborative relasionships to continue.
In particular, the close collaboration with the Kienberger Group of TU Munich/MPQ Garching Germany is often financially supported through the Bavaria California Technology Center (BaCaTeC) program \cite{BaCaTeC}.
