\subsection*{Organization of Major Activities}\addcontentsline{toc}{subsection}{Organization of Major Activities}
\begin{table}
\caption{\label{tab::schedule} Timeline of the R\&D schedule. Project Years (PY) are sub-divided into 6 month blocks. }
\begin{tabular}{l|l|l|l}
\hline
PY0.0 & PY0.5 & PY1.0 & PY1.5\\
\hline
somethign & else& and & this\\
somethign & else& and & this\\
\end{tabular}
\end{table}


\paragraph*{Period 0.0: 7/1/2018 -- 12/31/2018\\}
Test 2$\mu$m streaking with attosecond pulses from LCLS-I using Axial VMI \cite{Siqi2018}.\\
\textbf{Publication(s): }
``Attoclock Ptychography'' -- Tobias Schweizer \textit{et al.} \cite{Feurer2018}

\paragraph*{Period 0.5: 1/1/2019 -- 6/31/2019\\}
Design detectors for array, use existing data to develop algorithms for streaming analysis.\\
Implement on-board FPGA processing and demonstrate/simulate calibration learning, detail implementation plan for LCLS-II.\\
\textbf{Publication(s): }
``Online, single-shot characterization of few-femtosecond X-ray temporal pulse substructures at free-electron lasers via angular streaking'' -- Rupert Heider \textit{et al.}
``Machine Learning enabled x-ray pulse reconstruction'' --  Gregor Hartmann \textit{et al.}

\paragraph*{Period 1.0: 7/1/2019 -- 12/31/2019\\}
Construct detector array prototype for Building 40 benchmarks and LCLS-II beneficial occupancy.\\
\textbf{Publication(s): }
``Benchmarking/Novel algorithm for real-time x-ray pulse characterization at the LCLS-II.''

\paragraph*{Period 1.5: 1/1/2020 -- 6/31/2020, instrument commissioning\\}
Install array on LCLS-II and benchmark resolution with XAFS using 3 $\mu$m laser dressing of N$_2$O.
\textbf{Publication(s): }
``Streaming attosecond resolution x-ray pulse characterization: spectrum, polarization, and time.''
``Shaped attosecond x-ray FEL pulses for nonlinear x-ray science''
``Direct observation of laser-mixing of valence electronic symmetries.''

\subsection*{Responsibilities of key project personnel}\addcontentsline{toc}{subsection}{Responsibilities of key project personnel}
The PI will be responsible for leading the design effort to ensure a fundamentally integrated system, from chamber design and detector hardware to signal readout and waveform analysis.
He will also be principally concerned with the pulse reconstruction algorithm and data analysis pipeline.

Co-investigator Peter Walter is a Staff Scientist in LCLS Science Research and Development and will principally lead the detector simulations and construction.
Peter will also primarily lead the installation for early testing of the final concepts.
Co-investigator James Cryan is a Staff Scientist in LCLS Science Research and Development and PULSE and he will host much of the detector euv-based commissioning in his lab.
James will also continue to lead the VMI-based approach as well as many of the experimental simulations.
Both James and Peter will have regular interactions with the RA to help guide the work.

The RA will focus his or her on the design and integration of the chamber with the detector systems and will take responsibility for the detector construction.
He or she will organize and lead benchmark tests of attosecond angular streaking at other facilities such as EuroXFEL and in the Cryan lab in building 40.

We intend to recruit two graduate students from Stanford University to participate in the project.
We will seek funding through student research fellowships and LCLS-sponsored student outreach programs.
The PI has identified one such funded post-doctoral fellow, Audrey Therrien, who has needed expertise in charged particle simulations with GEANT4 as well as integrated circuit design.

Related research by the PI has attracted numerous externally funded post-doctoral fellows and visiting scientists, most notably Wofram Helml (Marie Curie Foundation), Anton Lidahl (Wallenberg Foundation), and Markus Ilchen (Volkswagen Foundation).
We expect these collaborative relationships to continue.
In particular, the close collaboration with the Kienberger Group of TU Munich/MPQ Garching Germany is often financially supported through the Bavaria California Technology Center (BaCaTeC) program \cite{BaCaTeC}.
