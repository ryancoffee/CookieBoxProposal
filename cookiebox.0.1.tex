%\documentclass[letterpaper,oneside,11pt,openany]{article}
\documentclass[letterpaper,oneside,11pt]{article}

\usepackage{anysize}%,layout}
\usepackage{url}
\usepackage{longtable}
\usepackage{amsmath}%\usepackage{matrix}
\usepackage{array}
\usepackage{booktabs}
\usepackage{lipsum}
\usepackage[dvipdfmx]{graphicx}
\usepackage{lscape}
\usepackage{verbatim}
\usepackage{cite}
\usepackage{listings}
\usepackage{geometry}
\usepackage{pdfpages}
\usepackage{wrapfig}
\usepackage{multicol}
\usepackage{fancyhdr}

\usepackage{enumitem}

%\usepackage{tgbonum}

 
\setlength{\headheight}{14pt}
\marginsize{22.5mm}{22.5mm}{22.5mm}{8mm}

\newcommand{\ket}[1]{\ensuremath{\left| #1 \right\rangle}}
\newcommand{\bra}[1]{\ensuremath{\left\langle #1 \right|}}
\newcommand{\braket}[2]{\ensuremath{\left\langle #1 \right|\left. #2 \right\rangle}}
\newcommand{\expect}[1]{\ensuremath{\left\langle #1 \right\rangle}}
\newcommand{\eV}{\mbox{eV}}
\newcommand{\icm}{\ensuremath{\mbox{cm}^{-1}}}
\newcommand{\cm}{\mbox{cm}}
\newcommand{\EE}[1]{\ensuremath{\times 10^{ #1 }}}
\newcommand{\ee}[1]{\ensuremath{\times 10^{ #1 }}}
\newcommand{\etal}{,~\textit{et~al.~}}
\newcommand{\norm}[1]{\ensuremath{\left| #1 \right|}}
\newcommand{\abs}[1]{\ensuremath{\left| #1 \right|}}
\newcommand{\HH}{\mathcal{H}}
\newcommand{\eps}{\varepsilon}
\newcommand{\degree}[1]{\ensuremath{#1^{\circ}}}

\renewcommand{\baselinestretch}{1}

\newcommand{\figwidth}{\linewidth}
\newcommand{\figheight}{0.5\linewidth}
\newcommand{\tabstop}{24pt}

\graphicspath{{./figs/}}

\title{Enabling long wavelength streaking for attosecond x-ray science.}
\author{
PI Ryan Coffee, Senior Staff Scientist\\
LCLS Science Research and Development \& The PULSE Institute,\\
Co-PI Peter Walter, Staff Scientist\\
LCLS Science Research and Development\\
SLAC National Accelerator Laboratory\\
650.387.0981, coffee@slac.stanford.edu\\
DOE National Laboratory Announcement Number: \textbf{FWP\#100498}\\
}

\setcounter{tocdepth}{3}

% from http://tex.stackexchange.com/questions/78221/changing-footnote-symbols

\makeatletter
\newcommand*{\myfnsymbolsingle}[1]{%
	\ensuremath{%
		\ifcase#1% 0
			\or % 1
			*%   
			\or % 2
			\dagger
			\or % 3  
			\ddagger
			\or % 4   
			\mathsection
			\or % 5
			\mathparagraph
			\else % >= 6
			\@ctrerr  
			\fi
	}%   
}   
\makeatother

\newcommand*{\myfnsymbol}[1]{%
	\myfnsymbolsingle{\value{#1}}%
}

% remove upper boundary by multiplying the symbols if needed
\usepackage{alphalph}
\newalphalph{\myfnsymbolmult}[mult]{\myfnsymbolsingle}{}

\renewcommand*{\thefootnote}{%
	\myfnsymbolmult{\value{footnote}}%
}




\begin{document}

%\thispagestyle{empty}
\pagestyle{fancy}
\lhead{Enabling long wavelength streaking for attosecond x-ray science}
%\lhead{Next generation electron spectrometers for attosecond XFEL science}
%\lhead{Next generation spectrometer array for attosecond XFEL science}
\rhead{Ryan N.~Coffee}

%\clearpage
\section*{Project Summary}
project_summary.1.1.tex

\end{document}


\pagebreak
\setcounter{page}{1}

\maketitle

\noindent{\Large \textbf{Project Narrative}}\\


\section*{Introduction}

introduction.1.1.tex




\section*{Objective \ref{obj::streaking} -- Streaking}
streaking.1.1.tex

\section*{Objective \ref{obj::euv-xray} -- euv-xray}
\input{active.euvxray.tex}



\begin{comment}
\section*{Budget \ref{sec::budget} -- budget}
year 1
\begin{enumerate}
\item RA (181)
\item AIR-TEC electronics engineer 115Hrs (25)
\item Vacuum Chamber (44)
\item 2x MCP electronics (15.8)
\item 1x 22k Digitizer + PCIe carrier, = 2x6GSps 
\item 4U computer (6K)
\item Xilinx Zynq UltraScale+ RFSoC ZCU111 Evaluation Kit (9)
\end{enumerate}

year 2
\begin{enumerate}
\item RA (188)
\item AIR-TEC electronics technician 120Hrs
\item 2x 22k Digitizer+PCIe carrier, gives a total of (3 digitizers total, each running 2x6GSps for total of 6 channels)
\item 4x MCP electronics (31.5)
\item FPGA for streaming analysis Virtex at 7k  
\item Miscellaneous vacuum and electronics M&S (14k)
\end{enumerate}
\end{comment}

%
%\begin{wrapfigure}[20]{r}{0.5\linewidth}
%\vspace{-.5\baselineskip}
%\centerline{
%	\includegraphics[trim = {-1.5cm -1.5cm 2.25cm 2cm},clip,width=\linewidth]{newcartoon/normal.cartoon.new.eps}
%}
%\vspace{-\baselineskip}
%\caption{
%	\label{spectralschematic} 
%	Normal, non-interferometric spectral encoding \cite{Bionta11,Harmand13,LemkeSPIE2013,MinaRSI}.
%	The supercontinuum pulse spans the visible spectrum and is chirped under normal dispersion such that the red end of spectrum arrives early while the blue end arrives late.
%	When an x-ray pulse intercepts the middle the pulse in a transparent dielectric material, the optical transmission is reduced for all colors that arrive after the x-ray pulse.
%	The ``head'' of this depletion edge indicates the arrival time of the x-ray pulse.
%}
%\end{wrapfigure}


\section*{Objective \ref{obj::real_time_analysis} -- Real time analysis}
\imput{active.realtimeanalysis.tex}





%This measurement will require a near \textit{in situ} implementation. %though it is fully compatible with feeding back into the synchronization system.
We propose to investigate a near \textit{in situ} interferometric version of spectrogram encoding, a hybrid of spectral and spatial encoding \cite{Nick2014}, to achieve 100 attosecond level delay resolution.
%A hybrid of spectral and spatial encoding, a two-dimensional spectrogram encoding method can provide even further time sorting resolution \cite{Nick2014}.
By crossing the chirped continuum beam relative to the x-ray beam in the material, we obtain a spatially-encoded arrival time in addition to the spectrally-encoded arrival time (Fig.~\ref{2dinterferesim}, spatial=vertical and spectral=horizontal). 
Multiple one-dimensional lineouts provide redundant measurements of the arrival time, e.g.~spectrally multiplexing the spatially encoded timing signal or vice versa.
Related to techniques used in de-noising images \cite{Candes2004a,Candes2004b,Candes2005,Elad2006}, the multiplexing provides an effective $\sqrt{N}$ improvement in the signal fidelity and thus a more accurately located edge.
This multiplexing makes for a more demanding sensitivity requirement than for the one dimensional system.
%The sensitivity requirement for the two-dimensional system is much more demanding than the one dimensional version.
%This makes the materials properties and thermal load particularly challenging for two-dimensional spectrogram encoding.
We will need to spread that original few pixel wide signal across the spatial dimension of the spectrograph to cover a few hundred pixel wide region.
In order to gain a factor of 10 in temporal resolution, one must multiplex in one of the two dimensions by a factor of 100.
This requires a 100 fold increase in signal compared to normal spectral encoding.
Here again interferometric sensitivity will help.


\section*{Objective \ref{obj::streaking} -- Attosecond angular streaking}



%%%%%%%%%%%%%%% TIMELINE %%%%%%%%%%%%%%%%%%%%%%%%%%%%%%% TIMELINE %%%%%%%%%%%%%%%%%%%%%%%%%%%%%%% TIMELINE %%%%%%%%%%%%%%%%


\subsection*{Organization of Major Activities\label{timeline}}
%\addcontentsline{toc}{subsection}{Organization of Major Activities}

We plan to have the base system ready for diagnosing the residual temporal jitter of LCLS-II in advance of the first light in mid-2019.
This will allow correlation with the linac and electron bunch parameters and will provide early debugging for the optical synchronization system.
The extended system is planned to become available in the early user runs of LCLS-II in 2020 for use in recovering sub-fs single-shot timing.
%We schedule the development of the attosecond angular streaking effort to coincide with the attosecond x-ray pulse developments at LCLS.
We will benchmark the resolution of the angular streaking method by measuring attosecond and multi-pulse, multi-color, multi-polarization mode of Ref.~\cite{Lutman2016} to benchmark our ability to directly measure the full time-dependent spectrum and polarization of novel shaped x-ray FEL pulses.


\paragraph*{Period 1: 7/15/2017 -- 7/14/2018, LCLS-I running\\}
\textbf{Objective \ref{obj::controlling}: } 
Test liquid sheets, graphene, perovskites, and scintillation samples, setup existing 2D opal and Princeton imaging spectrometer, build algorithm based on preliminary data and simulations, acquire 1D array detector and build data acquisition system.\\
\textbf{Objective \ref{obj::sorting}: }
Test birefringence versus pulse-front-tilt and imaging.\\
\textbf{Objective \ref{obj::streaking}: }
Test 3$\mu$m streaking with attosecond pulses from LCLS-I using DESY/Viefhaus CookieBox.\\
\textbf{Publication(s): }
``Interferometric Spectral Encoding for high rep-rate FELs''

\paragraph*{Period 2: 7/15/2018 -- 7/14/2019, LCLS-I down\\}
\textbf{Objective \ref{obj::controlling}: } 
Finalize the integrated optical system and algorithm and test with water, alcohol series, DMSO, and saline liquid sheets, finalize implementation matrix for various locations in LCLS-II.\\
\textbf{Objective \ref{obj::sorting}: }
Test test liquid sheet signal levels and polarization management, design matched spatial and spectral dispersion, and prototype algorithm dynamic ROI CMOS detection.\\
\textbf{Objective \ref{obj::streaking}: }
Design new CookieBox detector, use existing data to develop physics based algorithms for streaming analysis.\\
\textbf{Publication(s): }
``Femtosecond resolved non-equilibrium, ground electronic state, molecular dynamics in N$_2$O,'' and ``two-dimensional interferometric spectrogram encoded (2D-ISE) arrival time at high rep-rate FELs.''

\paragraph*{Period 3: 7/15/2019 -- 7/14/2020, LCLS-II commissioning\\}
\textbf{Objective \ref{obj::controlling}: } 
Commission CMOS lineout system on LCLS-II for synchronization feedback, measure the laser/x-ray jitter power spectrum and correlate with the machine parameters to debug weaknesses, benchmark performance and resolution with N$_2$O Auger electron spectra versus excited coherent bending and symmetric stretch vibrations.\\
\textbf{Objective \ref{obj::sorting}: }
Integrate dynamic ROI CMOS detector with on-board FPGA or equivalent smart hardware, develop the dynamic ROI algorithm.\\
\textbf{Objective \ref{obj::streaking}: }
Construct new modular CookieBox detector for LCLS-II.\\
\textbf{Publication(s): }
``Optical/x-ray pump-probe resolution at the LCLS-II facility,'' ``Time-domain lock-in amplification of weak interactions at the LCLS-II,'' and ``Novel algorithm for real-time x-ray pulse characterization at the LCLS-II.''

\paragraph*{Period 4: 7/15/2020 -- 7/14/2021, LCLS-II operating\\}
\textbf{Objective \ref{obj::sorting}: }
Implement on-board FPGA processing and demonstrate CMOS non-linear ROI masking, build implementation matrix for various locations in LCLS-II.\\
\textbf{Objective \ref{obj::streaking}: }
Build passive CEP locked 3 $\mu$m pulses, commission new CookieBox detector for LCLS-II, develop algorithm for streaming analysis, benchmark 150 attosecond resolution.\\
\textbf{Publication(s): }
``Streaming attosecond resolution x-ray pulse characterization: spectrum, polarization, and time.''
``Optically shaped attosecond x-ray FEL pulses for nonlinear x-ray science''

\paragraph*{Period 5: 7/15/2021 -- 7/14/2022, LCLS-II operating\\}
\textbf{Objective \ref{obj::controlling}: }
Plan extension to hard x-ray regime.\\
\textbf{Objective \ref{obj::sorting}: }
Commission full system with CMOS detector and on-board processing.\\
\textbf{Objective \ref{obj::streaking}: }
Install on LCLS-II with on-board processing, benchmark resolution with NEXAFS of 3 $\mu$m laser dressing of N$_2$O, and develop the implementation matrix for various locations in LCLS-II, and explore high rep-rate application to hard x-ray regime.\\
\textbf{Publication(s): }
``Direct observation of laser-mixing of valence electronic symmetries.''


\subsection*{Responsibilities of key project personnel\label{personnell}}
%\addcontentsline{toc}{subsection}{Responsibilities of key project personnel}

The PI will be responsible for organizing the major efforts including algorithm development, optical design, and coordination of engineering design and hardware construction.
He is expected to contribute 65\% of his effort to this project.

The first RA/Project Scientist will be hired at the 75\% level with 25\% of his or her effort will be shared with closely related Accelerator R\&D projects.
He or she will work primarily on the optical construction and execution of the experiments in the first two years for the interferometric timing effort.
In addition, he or she will organize an initial preliminary benchmark test of attosecond angular streaking based on the Viefhaus CookieBox that will be borrowed from DESY.
He or she will begin the effort of designing the CookieBox detector that will be used in the attosecond angular streaking effort.
A second RA/Project Scientist will take over the CookieBox detector construction and will lead the attosecond streaking beamtime preparations in years 3-5.
He or she will focus also on the optical design and fabrication responsibilities for the final dedicated CookieBox detector and required laser implementation.

We intend to recruit two graduate students from Stanford University to participate in the project.
We will seek funding through student research fellowships and LCLS-sponsored student outreach programs.
Related research by the PI has attracted numerous externally funded post-doctoral fellows and visiting scientists, most notably Wofram Helml (Marie Curie Foundation), Anton Lidahl (Wallenberg Foundation), and Markus Ilchen (Volkswagen Foundation).
We expect these collaborative relasionships to continue.
In particular, the close collaboration with the Kienberger Group of TU Munich/MPQ Garching Germany is often financially supported through the Bavaria California Technology Center (BaCaTeC) program \cite{BaCaTeC}.



%%%%%%%%%%%%%%%% BIO SKETCH %%%%%%%%%%%%%%%%%%%%%%%%%%%%%%%%% BIO SKETCH %%%%%%%%%%%%%%%%%%%%%%%%%%%%%%%%% BIO SKETCH %%%%%%%%%%%%%%%%%


\clearpage
\appendix

\section*{Appendix 1: Biographical Sketch}
\addcontentsline{toc}{section}{Appendix 1: Biographical Sketch}
\subsection*{Ryan Coffee}
LCLS Laser Department\hfill Tel:650.387.0981\\
SLAC National Accelerator Laboratory\hfill Fax:650.926.2521\\
Mail Stop 20, Menlo Park, California  94025\hfill E-mail: coffee@slac.stanford.edu\\

\subsection*{Education and Training}
%\addcontentsline{toc}{subsection}{Education and Training}
\begin{tabular}{llll}
Research Associate & & SLAC National Accelerator Laboratory & 06/2006--04/2009\\
Ph.D. & Physics & University of Connecticut & 06/2006\\
M.S. & Physics & University of Connecticut & 12/2001\\
B.S & Physics & University of Arkansas & 06/1999\\
B.A. & Philosophy & University of Arkansas & 06/1999
\end{tabular}

\subsection*{Research and Professional Experience}
%\addcontentsline{toc}{subsection}{Research and Professional Experience}
\textbf{01/2014--present Staff Scientist}, PULSE Institute\\
\textbf{04/2009--present Staff Scientist}, LCLS Laser Division, SLAC\\
Spectral and spectrogram encoding of relative x-ray arrival time, 
sub-10 fs pulse generation for FEL multiplicative seeding and for time resolved photo-chemistry, 
optical and THz laser streaking techniques at the LCLS, 
angle-resolved double- and single-core hole spectroscopy of impulsively-aligned molecules, 
x-ray pump/x-ray probe experiments at LCLS, 
x-ray pulse shaping for multi-dimensional x-ray spectroscopy, 
gas phase ultrafast electron diffraction, 
LCLS experimental laser facility installation and commissioning\\
\textbf{06/2006--04/2009 Research Associate}, PULSE Institute\\
Coherent control of rotational wave-packet motion in ambient nitrogen and iodine.\\
\textbf{01/2006--06/2006 Research Associate}, University of Michigan\\
Participation in two of the final SPPS experiments\\
\textbf{09/1999--06/2006 Research Assistant}, Department of Physics, University of Connecticut\\
Two-color pump-probe optical experiments with nitrogen, 
molecular vibrational wave-packet motion on laser induced potential energy surfaces, 
ion time-of-flight spectroscopy, 
vuv-fluorescence spectroscopy of selective high-order multi-photon absorption in N$_2$, 
transient absorption spectroscopy.

\setlist{nosep}%,after=\vspace{\baselineskip}}

\subsection*{Selected publications}
\addcontentsline{toc}{subsection}{Selected publications}
\begin{enumerate}
\item \textit{Optical Shaping of X-Ray Free-Electron Lasers}
A Marinelli, \textbf{R Coffee}\etal
Physical Review Letters, \textbf{116}, 254801 (2016)
\item \textit{Polarization control in an X-ray free-electron laser}
AA Lutman \dots \textbf{R Coffee}\etal 
Nature Photonics, \textbf{10}, 468 (2016)
\item \textit{Generating femtosecond X-ray pulses using an emittance-spoiling foil in free-electron lasers}
Y Ding, C Behrens, \textbf{R Coffee}\etal
Applied Physics Letters \textbf{107}, 191104 (2015)
\item \textit{High-intensity double-pulse X-ray free-electron laser}
A Marinelli, \ldots \textbf{R Coffee}\etal
Nature Communications \textbf{6}, 6369 (2015)
\item \textit{Measuring the temporal structure of few-femtosecond FEL X-ray pulses directly in the time domain}
W Helml, \ldots \textbf{R Coffee}\etal
Nature Photonics, \textbf{8}, 950 (2014)
\item \textit{Sub-femtosecond precision measurement of relative X-ray arrival time for free-electron lasers}
N Hartmann,\etal \textbf{RN Coffee}
Nature Photonics \textbf{8}, 706 (2014)
\item \textit{Spectral encoding method for measuring the relative arrival time between x-ray/optical pulses}
M Bionta\etal \textbf{R Coffee}
Review of Scientific Instruments, \textbf{85}, 083116 (2014)
\item \textit{Multicolor Operation and Spectral Control in a Gain-Modulated X-Ray Free-Electron Laser}
A Marinelli,\etal\ldots \textbf{RN Coffee}, and C Pellegrini
Physical Review Letters \textbf{111}, 134801 (2013)
\item \textit{Experimental demonstration of femtosecond two-color x-ray free-electron lasers}
AA Lutman, R Coffee\etal
Physical Review Letters \textbf{110}, 134801 (2013)
\item \textit{Spectral encoding of x-ray/optical relative delay}
Mina R. Bionta,\etal \textbf{R. N. Coffee}
Optics Express, \textbf{19}, 21855 (2011)
\end{enumerate}


\subsection*{Synergistic Activities}
%\addcontentsline{toc}{subsection}{Synergistic Activities}

\paragraph*{Ultrafast electron diffraction (UED)}
%\addcontentsline{toc}{subsubsection}{Ultrafast electron diffraction (UED)}
The PI has recently become intimately involved with the use of ultrafast electron diffraction (UED) in order to merge spectroscopic studies at the LCLS with the structural sensitivity of electron diffraction \cite{ued_rsi2015,Jie2016}.
The merger of the two experimental paradigms is enabled by induced time-domain coherent molecular motions that in-turn produce concerted fluctuations in both x-ray spectra and in electron diffraction features.
By pattern recognizing these common-mode fluctuations, one can merge experimental results as if the two experiments were performed simultaneously.
The PI is therefore keenly attuned to need for exquisite synchronization at high energy ultrafast electron diffraction sources.
The successful outcome of Objective~\ref{obj::controlling} would be immediately applicable to high energy electron diffraction facilities.% such as SLAC's UED facility, LBNL's facility, and others.


\paragraph*{Deep ultra-violet}
During period two, the LCLS facility will be down.
Here we will draw on our existing research program investigating routes to broad-band tunable deep uv.
This collaborative work together with the Keinberger group of TU Munich produced a Marie-Curie Fellowship project of Wolfram Helml to develop ultra-broadband, sub-10 fs duration deep ultraviolet pulses.
This project later led to a Masters student research project of Patrick Rupprecht to develop the required pulse characterization in the deep uv regime.
This deep ultraviolet will be used as a surrogate for the x-ray pulse during the LCLS down time from August 2018 -- August 2019.
This existing setup will continue to serve as a surrogate source for testing and debugging new concepts and materials. 

%\paragraph*{Stanford Medical}
%\addcontentsline{toc}{subsubsection}{Stanford Medical}
%The PI has recently opened a new collaboration with the group of Dr.~Craig Levin in the Division of Nuclear Medicine at Stanford.
%Through this collaboration we expect a mutually beneficial investigation into novel scintillation materials for improving the cascade time and the signal per dose ratio.
%Such a time resolution would greatly improve the spatial resolution of the PET scan per unit time spent undergoing the scan. 
%Ideally one could imagine then making such scans in short time-series if each scan only takes a few seconds.
%Then one can observe the dynamic processes in e.g.~brain activity.


\paragraph*{X-ray pulse shaping}
%\addcontentsline{toc}{subsubsection}{X-ray pulse shaping}
The PI has played a central role in motivating and helping the development of the many electron bunch based methods for x-ray pulse shaping \cite{Lutman13_twocolor,Marinelli13_twocolor,Helml2014,Marinelli2015,Lutman2016,Marinelli2016}.
From performing the first double-slotted foil experiment for two x-ray pulses \cite{CoffeeDAMOP11} to demonstrating optical carving of the electron bunch \cite{Marinelli2016} and helping demonstrate multi-polarization multi-color operation \cite{Lutman2016}, our group has helped to pioneer many of the pulse shaping schemes that have been developed thus far at the LCLS.
On the active forefront of pulse shaping, he is uniquely positioned to develop the pulse shape diagnostics hand-in-hand with the accelerator R\&D for which he is already an active contributor.
This close overlap with the LCLS Accelerator R\&D group is in fact what allows for the shared resources such as the two Project Scientist/RA positions to be shared so intimately.
These positions will be fundamentally interwoven between the development of novel FEL methods and the attosecond diagnostics required to interrogate those methods.


\paragraph*{Machine learning}
%\addcontentsline{toc}{subsubsection}{Machine learning}

Such so-called ``deep learning'' is used to interrogate the output of the artificial neural networks to check the physical interpretation of how the output is determined \cite{Mihir}.
Our preliminary test, shown in Fig.~\ref{mlsorted}, uses the neural network classifier to sort LCLS shots based on a CNN predicted two-color spectral distribution.
The sorted shots are shown at right and an example of an XTCAV image is shown in the top left of Fig.~\ref{mlsorted}.
Guided back propagation is then used to interrogate the neural network about which regions of that image influenced the sorting most strongly, and this is shown in the lower left.
These two regions indeed indicate the loss of electron energy due to the lasing process, thus boosting our confidence that the CNN is sorting based on valid physical properties of the x-ray pulse, photon energy separation in this case.

%\begin{comment}
\begin{wrapfigure}[18]{r}{.6\linewidth}
\vspace{-1\baselineskip}
\centerline{
\includegraphics[height=7cm]{plotting.both.eps}
\includegraphics[height=7cm]{sorted.lower.fixed.upper.eps}
}
\caption{\label{mlsorted}(left) Using machine learning to identify subtle features in 
(right) An example of sorting molecular Auger electron spectra based on CNN predicted FEL two-color spectrum from XTCAV. }
\end{wrapfigure}
%\end{comment}

\subsection*{List of collaborators and co-authors: (48 months)}

\vspace{1\baselineskip}
\noindent\textbf{Collaborators:}\\
\vspace{-1.5\baselineskip}
{\small
\begin{multicols}{2}
\noindent 
	Lorenzo Avaldi \hfill CNR-ISM, Rome\\ 
	Nora Berrah \hfill Univ. of Connecticut\\
	Martin Beye \hfill HZB Berlin\\
	Christoph Bostedt \hfill ANL\\
	Marco Cammarata \hfill Univ. of Rennes, France\\
	Adrian Cavalieri \hfill CFEL Hamburg\\
	Martin Centurion \hfill U. Nebraska,\\
	Tilo Doeppner \hfill LLNL\\
	Gilles Doumy \hfill ANL \\
	Stefan D\"usterer \hfill FLASH DESY Hamburg\\
	Raimund Feifel \hfill Univ. of Gothenburg, Sweden\\
	Thomas Fennel \hfill Univ. Rostock, Germany
	Thomas Feurer \hfill Univ. of Bern, Switzerland\\
	Thornton Glover \hfill Gordon \& Betty Moore Found.\\
	Jan Gr\"unert \hfill Euro. XFEL, Hamburg\\
	Markus G\"uhr \hfill Potsdam University, Germany\\
	Marion Harmand \hfill IMPMC-UPMC, Paris, France\\
	Janos Hajdu \hfill Uppsala Univ. Sweden\\
	Christoph Hauri \hfill SwissFEL PSI, Switzerland\\
	Dan Kane \hfill Mesa Photonics, Albequerque\\
	Reinhard Kienberger \hfill TU Munich\\
	Jochen K\"upper \hfill CFEL, Hamburg\\
	Jerry LaRue \hfill Chapman, Irvine CA\\
	Jon Marangos \hfill Imperial College, UK\\
	Marc Messerschmidt \hfill BioXFEL, Hamburg\\
	Michael Meyer \hfill Euro. XFEL, Hamburg\\
	Catalin Miron \hfill ELI-Delivery Consortium\\
	Thomas M\"oller \hfill TU Berlin\\
	Serguei Molodtsov \hfill Euro XFEL, Hamburg\\
	Anders Nilsson \hfill Uppsala University, Sweden\\
	Steve Pratt \hfill ANL\\
	Artem Rudenko \hfill Kansas State University\\
	Daniel Rolles \hfill Kansas State University\\
	Nina Rohringer \hfill U. of Hamburg, Germany\\
	Arnaud Rouzee \hfill MBI Berlin\\
	Ilme Schlichting \hfill MPI Heidelberg\\
	Sharon Shwartz \hfill Bar-Ilan University, Israel\\
	Klaus Sokolowski-Tinten \hfill U. of Duisburg-Essen,\\\mbox{ } \hfill Essen Germany\\
	Thomas Tschentscher \hfill Euro. XFEL Hamburg\\
	Kiyoshi Ueda \hfill Tohoku Univ., Japan\\
	Joachim Ullrich \hfill PTB Germany\\
	Jens Viefhaus \hfill DESY
\end{multicols}
}
\normalsize
\subsection*{Graduate and Postdoctoral Advisors}
G.~Gibson (University of Connecticut), P.H.~Bucksbaum (PULSE/Stanford)


















\clearpage
\appendix
\section*{Appendix 2: Current and Pending Support}
\addcontentsline{toc}{section}{Appendix 2: Current and Pending Support}
Both current and pending support will be predominantly covered under the U.S. Department of Energy / Stanford University Contract for Management and Operation of SLAC National Accelerator Laboratory with a small fraction under the National Institute of Health.\\
\vspace{\baselineskip}\\
\begin{tabular}{lll}
Current Support & LCLS-Soft X-ray Department & 50\%  \\
		& LCLS High Sensitivity Timing & 20\% \\
		& LDRD Machine Learning for LCLS-II & 20\%  \\
		& NIH Time-of-Flight PET & 10\% \\
		\vspace{2\baselineskip} &&\\
Pending support&same as current & \\
\end{tabular}











\clearpage
\appendix
\section*{Appendix 3: Bibliography and References Cited}% Bibliography and Refrences Cited}
\addcontentsline{toc}{section}{Appendix 3: Bibliography and References Cited}

\bibliographystyle{unsrt}
\bibliography{$BIBFILES/hhg,$BIBFILES/medicine,$BIBFILES/mariano,$BIBFILES/collaborations,$BIBFILES/polarization,$BIBFILES/vmi,$BIBFILES/timing,$BIBFILES/carriers,$BIBFILES/rixs,$BIBFILES/ued,$BIBFILES/aromatic,$BIBFILES/lcls_refs,$BIBFILES/shaping,$BIBFILES/streaking,$BIBFILES/supercontinuum,$BIBFILES/fel_multicolor,$BIBFILES/attosecond,$BIBFILES/Coffee,$BIBFILES/xrayspectroscopy,$BIBFILES/computing,$BIBFILES/cookiebox,$BIBFILES/Bucksbaum,$BIBFILES/time-frequency,$BIBFILES/deepuv,$BIBFILES/Gibson,$BIBFILES/materials}









\clearpage
\appendix
\section*{Appendix 4: Facilities and Other Resources}
\addcontentsline{toc}{section}{Appendix 4: Facilities and Other Resources}
%\subsection*{Facilities}
%\addcontentsline{toc}{subsection}{Facilities}

There are two principle lab spaces identified for this project.
\begin{itemize}
\item LCLS-AMO laser system when not in use by user experiments, this is the primary laser for the x-ray spectroscopy arm of this program.
\item Laser only based commissioning and development experiments will be carried out in one of the new laser labs in the Photon Science Laboratory Building (PSLB).
\end{itemize}

Analysis resources exist from both the SLAC-unix farm and the LCLS-unix farms.  
Our long collaboration with the data analysis and controls groups at LCLS not only allows the PI particularly early insight into the computing resources, but is also motivates his active pursuit of data compression and on-board analysis algorithms.
This mutual benefit ensures the continued use and support for these computing facilities.

Office space will be available for all team members in the Laser Division area of building 751 with additional space for students and visiting scientists in the PULSE Institute.

\subsection*{Additional Personnel}
\addcontentsline{toc}{subsection}{Additional Personnel}

As noted above, there will be likely two graduate student recruited from Stanford University who will work on this project.
Those students participate via the student outreach programs of LCLS at SLAC.




\clearpage
\appendix
\section*{Appendix 5: Equipment}
\addcontentsline{toc}{section}{Appendix 5: Equipment}

\noindent The primary laser system identified for offline optical work will reside in one of the new laser labs located in the Photon Science Laboratory Building (PSLB) at SLAC.
Prior to the occupancy of PSLB, we will focus our efforts on using the AMO laser system at LCLS for both beamtime preparations as well as offline work when not interfering with scheduled user experiments.
Required equipment under the general use of the laser division at LCLS are:
\begin{itemize}
\item Noble-gas-cell based continuum generation and pulse compression system.
\item Visible imaging spectrometer for spectrogram retrieval.
\item Data Acquisition System including high frame-rate linear CCD and 2D CMOS cameras
\end{itemize}
\vspace{\baselineskip}
with associated laser systems: 
\begin{itemize}
\item LCLS-AMO laser system is available when not serving user experiments and during the projected LCLS down times.
\item The R\&D laser lab in the PSLB will house an amplified Ti:Sapphire laser system similar to the existing LCLS laser systems.
\item 
\end{itemize}
\vspace{\baselineskip}

\noindent Other equipment available to the project include:
TOPAS-c 2 for generation of 3 $\mu$m pulses.

\clearpage
\appendix
\section*{Appendix 6: Data Management Plan}
\addcontentsline{toc}{section}{Appendix 6: Data Management Plan}

As stated in the project narrative, one of the central themes is to reduce the data load.
The data that is accumulated as part of this project will be made broadly available both internally via SLAC/LCLS unix user account access and, by inquiry to the PI, externally via coordinated data formatting and access FTP.

Data taken with the LCLS will be stored in accordance with LCLS policy also in SLAC Central Storage (or LCLS storage if in future LCLS moves the storage service).
Access can then be granted by the PI also for any individual who obtains an LCLS user unix account. 

The laser lab based data will be housed in the SLAC Central Computing.  
The PI currently maintains a 1 TB/year subscription and that will be incrementally increased up to a 5 TB/year storage, expanded when needed.
%The subscription cost is included in the budget.
The data will be made broadly available by SLAC unix account access and externally by contacting the PI and coordinating an FTP service of the data.


\clearpage
\appendix
\section*{Appendix 7: Letters of Collaboration}
\addcontentsline{toc}{section}{Appendix 7: Letters of Collaboration}

Please see attached Letters of Collaboration.

\end{document}


%%%%%%%%%%%%%%%%%%%%%%%%%%%% END DOCUMENT %%%%%%%%%%%%%%%%% END DOCUMENT %%%%%%%%%%%%%%%%%%%%%%



