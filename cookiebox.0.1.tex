%\documentclass[letterpaper,oneside,11pt,openany]{article}
\documentclass[letterpaper,oneside,11pt]{article}

\usepackage{anysize}%,layout}
\usepackage{url}
\usepackage{longtable}
\usepackage{amsmath}%\usepackage{matrix}
\usepackage{array}
\usepackage{booktabs}
\usepackage{lipsum}
\usepackage[dvipdfmx]{graphicx}
\usepackage{lscape}
\usepackage{verbatim}
\usepackage{cite}
\usepackage{listings}
\usepackage{geometry}
\usepackage{pdfpages}
\usepackage{wrapfig}
\usepackage{multicol}
\usepackage{fancyhdr}

\usepackage{enumitem}

%\usepackage{tgbonum}

 
\setlength{\headheight}{14pt}
\marginsize{22.5mm}{22.5mm}{22.5mm}{8mm}

\newcommand{\ket}[1]{\ensuremath{\left| #1 \right\rangle}}
\newcommand{\bra}[1]{\ensuremath{\left\langle #1 \right|}}
\newcommand{\braket}[2]{\ensuremath{\left\langle #1 \right|\left. #2 \right\rangle}}
\newcommand{\expect}[1]{\ensuremath{\left\langle #1 \right\rangle}}
\newcommand{\eV}{\mbox{eV}}
\newcommand{\icm}{\ensuremath{\mbox{cm}^{-1}}}
\newcommand{\cm}{\mbox{cm}}
\newcommand{\EE}[1]{\ensuremath{\times 10^{ #1 }}}
\newcommand{\ee}[1]{\ensuremath{\times 10^{ #1 }}}
\newcommand{\etal}{,~\textit{et~al.~}}
\newcommand{\norm}[1]{\ensuremath{\left| #1 \right|}}
\newcommand{\abs}[1]{\ensuremath{\left| #1 \right|}}
\newcommand{\HH}{\mathcal{H}}
\newcommand{\eps}{\varepsilon}
\newcommand{\degree}[1]{\ensuremath{#1^{\circ}}}

\renewcommand{\baselinestretch}{1}

\newcommand{\figwidth}{\linewidth}
\newcommand{\figheight}{0.5\linewidth}
\newcommand{\tabstop}{24pt}

\graphicspath{{./figs/}}

\title{Enabling long wavelength streaking for attosecond x-ray science.}
\author{
PI Ryan Coffee, Senior Staff Scientist\\
LCLS Science Research and Development \& The PULSE Institute,\\
Co-investigator Peter Walter, Staff Scientist\\
LCLS Science Research and Development\\
SLAC National Accelerator Laboratory\\
650.387.0981, coffee@slac.stanford.edu\\
DOE National Laboratory Announcement Number: \textbf{FWP\#100498}\\
}

\setcounter{tocdepth}{3}

% from http://tex.stackexchange.com/questions/78221/changing-footnote-symbols

\makeatletter
\newcommand*{\myfnsymbolsingle}[1]{%
	\ensuremath{%
		\ifcase#1% 0
			\or % 1
			*%   
			\or % 2
			\dagger
			\or % 3  
			\ddagger
			\or % 4   
			\mathsection
			\or % 5
			\mathparagraph
			\else % >= 6
			\@ctrerr  
			\fi
	}%   
}   
\makeatother

\newcommand*{\myfnsymbol}[1]{%
	\myfnsymbolsingle{\value{#1}}%
}

% remove upper boundary by multiplying the symbols if needed
\usepackage{alphalph}
\newalphalph{\myfnsymbolmult}[mult]{\myfnsymbolsingle}{}

\renewcommand*{\thefootnote}{%
	\myfnsymbolmult{\value{footnote}}%
}




\begin{document}

\pagestyle{fancy}
\lhead{Enabling long wavelength streaking for attosecond x-ray science}
\rhead{Ryan N.~Coffee}

%\clearpage
\setcounter{page}{0}

\thispagestyle{empty}
\section*{Project Summary}\addcontentsline{toc}{section}{Project Summary}
project_summary.0.3.tex

\maketitle
\thispagestyle{empty}
\tableofcontents

\pagebreak
\pagestyle{fancy}
\setcounter{page}{1}

\noindent{\Large \textbf{Project Narrative}}\\

\section*{Introduction}\addcontentsline{toc}{section}{Introduction}
introduction.1.0.tex

\subsection*{Streaking for Xray Pulse Reconstruction}\addcontentsline{toc}{subsection}{Streaking for X-ray Pulse Reconstruction}
streaking.1.0.tex

\subsection*{Time-resolved photo and Auger electron emission}\addcontentsline{toc}{subsection}{Time-resolved photo and Auger electron emission}
photoauger.0.0.tex

\subsection*{Attosecond EUV/x-ray pump-probe experiments}\addcontentsline{toc}{subsection}{Attosecond EUV/x-ray pump-probe experiments}
euvxray.0.0.tex

\section*{Objectives}\addcontentsline{toc}{section}{Objectives}

\subsection*{Optimized Detector Array}\addcontentsline{toc}{subsection}{Optimized Detector Array}
detectordetails.1.0.tex

\subsection*{Real time analysis}\addcontentsline{toc}{subsection}{Real time analysis}
realtimeanalysis.0.0.tex


%%%%%%%%%%%%%%% TIMELINE %%%%%%%%%%%%%%%%%%%%%%%%%%%%%%% TIMELINE %%%%%%%%%%%%%%%%%%%%%%%%%%%%%%% TIMELINE %%%%%%%%%%%%%%%%
\section*{Schedule}\addcontentsline{toc}{section}{Schedule}
schedule.0.0.tex


\clearpage
\appendix

%%%%%%%%%%%%%%%% BIO SKETCH %%%%%%%%%%%%%%%%%%%%%%%%%%%%%%%%% BIO SKETCH %%%%%%%%%%%%%%%%%%%%%%%%%%%%%%%%% BIO SKETCH %%%%%%%%%%%%%%%%%
\section*{Appendix 1: Biographical Sketch}\addcontentsline{toc}{section}{Appendix 1: Biographical Sketch}
biosketch.0.0.tex

\clearpage
append2-6.1.0.tex

%
%\clearpage
%\appendix
%\section*{Appendix 7: Letters of Collaboration}\addcontentsline{toc}{section}{Appendix 7: Letters of Collaboration}
%
%Please see attached Letters of Collaboration.

\end{document}


%%%%%%%%%%%%%%%%%%%%%%%%%%%% END DOCUMENT %%%%%%%%%%%%%%%%% END DOCUMENT %%%%%%%%%%%%%%%%%%%%%%



