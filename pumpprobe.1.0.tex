\begin{wrapfigure}[9]{r}{3in}
\vspace{-3\baselineskip}
\centerline{\includegraphics[width=\linewidth]{naturePhoton_NickWolfi.doublepulses.eps}}
\vspace{-0.5\baselineskip}
\caption{\label{fig::doublepulses}Reproduced from Ref.~\cite{Nick2018}.
}
\end{wrapfigure}
Many attosecond scale experiments are currently only enabled by high-harmonic generation (HHG)\cite{Lewenstein1994,Hentschel2001,Chen2014,Biegert2014,Schmidt2016,Biegert2016,WornerSci2017,Zenghu2017}.
These experiments could greatly benefit from the much higher brightness of an attosecond xFEL beamline \cite{Ding2009,Xiang2009,xLEAP}, effectively exchanging the flux challenge of the HHG sources for the synchronization challenge of xFEL pulses.
With the ability for x-ray pulse characterization at the attosecond level, one removes the synchronization challenge for x-ray pump/x-ray probe experiments.
Figure~\ref{fig::doublepulses} indeed shows that angular streaking can not only identify double pulses, but also sort such pulses into relative delay \cite{Nick2018}.
Roughly 1\% of the SASE pulses measured consisted of only two spikes when running in low charge mode with emittance shaping \cite{EmmaFoil}.
Such pulses can then be sorted, allowing a ``measure-and-sort'' x-ray pump/x-ray probe experimental paradigm.

Chirality dynamics was also featured as a major scientific thrust area for LCLS-II \cite{lcls2_opportunities}.
One can imagine using one color of x-ray pulse to pump a chiral molecule such as trifluoromethyloxirane \cite{Ilchen2017} at a fluorine or cabon atom and then probe the evolution of the valence electronic response via XAFS at the oxygen edge.
In this way, one would prefer to use likely different polarizations for pulse-pairs whereby one pulse is tuned to the oxygen edge and the other is tuned to the carbon or fluorine edge.
Since such a capability is enabled by the combination of Delta undulator for polarization control together with the variable gap soft x-ray undulator of LCLS-II in split mode, we also require that our attosecond pulse diagnostic allows for also polarization diagnosis pulse pairs separated by well over to 200~eV.

If instead, one prefers a weaker isolated attosecond probe pulse, one could use a more traditional laser-based HHG source.
Such pulses in the euv could be used to pump inner valence transitions in molecular systems, setting up rather pronounced coherent electronic motions, or simply used as broad-band transient absorption probes \cite{Biegert2016}.
Then the attosecond x-ray pulses from the LCLS could be used to interrogate the valence occupation via time-resolved photo and Auger electron spectroscopy.
In this case, the euv light is sufficient to produce attosecond bursts of electrons from a helium buffer gas while the x-ray pulses would equally well produce high energy photoelectrons also from helium.
In this way, angular streaking could simultaneously recover, on a single shot basis, the relative delay between x-ray/x-ray and even euv/x-ray attosecond pulse pairs.

