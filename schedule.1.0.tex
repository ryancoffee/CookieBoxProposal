%\begin{table}
%\caption{\label{tab::schedule} Timeline of the R\&D schedule. Project Years (PY) are sub-divided into 6 month blocks. }
\small
\begin{tabular}{l|l|l|l}
\hline
\multicolumn{2}{c|}{Project Year 1} &  \multicolumn{2}{c}{Project Year 2} \\
\hline
eTOF spectrometer design & Simulate final design & Test inital MCPs & Test Final system\\
Initial MCP electronics & Install initial MCPs & Complete MCP set & Final install\\
Chamber Design & Purchase Chamber& Install inital spectrometers & Complete Chamber\\
Initial Digitizers & Simulate waveform sampling& Complete Digitizer set & Minimize cross-talk\\
\hline
Purchase SystemOnChip & & Purchase FPGA &\\
Develop algorithms& Train ML on sims & Transfer Learning & Benchmark latency\\
\hline
\end{tabular}
%\end{table}
\normalsize

\paragraph*{Project Year 1}
In the beginning of Project Year 1, two prototype detector electronics, the MCP assemblies, will be ordered.
The bench testing of these two detectors will test for cross-talk interactions and ringing.
We will also purchase LCLS-II compatible digitizer electronics that represent an upgrade to the current expectations.
These digitizers will be chosen based on the simulated waveforms and the measured impulse response of the MCP assemblies.
The signal testing will also provide further insight into the actual signal processing that will be needed in the final build out.  
The vacuum chamber chamber design will be finalized for purchasing. 
The detectors will be installed into the vacuum chamber.
Since fringe electric fields may compromise the final electron energy resolution, along with stray magnetic fields, we must test the prototype detector pair for cross-talk and ringing in the final chamber environment.  
\begin{itemize}
\item Test 2$\mu$m streaking with attosecond pulses from LCLS-I using Axial VMI \cite{Siqi2018}.
\item Design detectors for array, use existing data to develop algorithms for streaming analysis.
\item Purchase all electronics supplies and chamber.
\item Design transfer learning implementation plan for LCLS-II.
\item Test waveform sampling versus pulse reconstruction accuracy.
\end{itemize}

\paragraph*{Project Year 2}
In the beginning of Project Year 2, we will partially complete the detector array with an additional 4 detectors assemblies with corresponding digitizers.
With a total of 6 modules, we will be able to test the multi-retardation window scheme and its impact on also measuring a third, Auger channel.
This will allow a demonstration of the design performance for simultaneous reconstruction of two-color x-ray pulses along with time-resolved Auger electron spectroscopy.
We will use the 120~Hz LCLS-I machine running into the soft x-ray undulators to demonstrate the spectral range of the detectors and the resolution.  
We will also develop the calibration protocol which will then be automated as the system becomes integrated into the LCLS-II controls system and DAQ.  
We note that expected configuration of an ML-enabled data reduction node is currently being designed with this detector as one of the use-cases for on-board streaming analysis.  
We will therefor interface directly with the group of Jana Thayer to ensure full compatibility with the LCLS-II Data Reduction Pipeline (DRP).  
By August of 2020, we expect the system would be ready for full deployment, just in time to receive LCLS-II, high repetition rate pulses.
\begin{itemize}
\item Implement on-board FPGA processing and demonstrate calibration learning.
\item Optimize FPGA usage and benchmark SoC solution compared to larger ML-server solution.
\item Construct detector array prototype for Building 40 benchmarks and LCLS-II beneficial occupancy.
\item Install array on LCLS-II and benchmark resolution with XAFS using 3 $\mu$m laser dressing of N$_2$O.
\end{itemize}

\paragraph*{Responsibilities of key project personnel}\addcontentsline{toc}{paragraph}{Responsibilities of key project personnel}
The PI will be responsible for leading the design effort to ensure a fundamentally integrated system, from chamber design and detector hardware to signal readout and waveform analysis.
He will also be principally concerned with the pulse reconstruction algorithm, inference engine, and data analysis pipeline.

Co-investigator Peter Walter is a Staff Scientist in LCLS Science Research and Development and will principally lead the detector simulations and construction.
Peter will also primarily lead the installation for early testing of the final concepts.
Co-investigator James Cryan is a Staff Scientist in LCLS Science Research and Development and PULSE and he will host much of the detector euv-based commissioning in his lab.
James will also continue to lead the VMI-based approach as well as many of the experimental simulations.
Both James and Peter will have regular interactions with both the PI and the RA to help inform and guide the work.

The Research Associate (RA) will spend 100\% of his or her time over the 2-year proposal period.  
In Year 1, he or she will use existing computational infrastructure to simulate the electron Time-of-Flight detector array, iterating the design in order to minimize the cross-talk between eTOF spectrometers in the array.  
Furthermore, he or she will identify and order the electron multiplier assemblies and digitizer electronics that work together to optimize energy resolution and spectral window.  
Upon reciept, the RA will complete the initial 2 prototype eTOF detectors and begin the test/iterate cycle.
In Year 2, the RA will focus on completing construction of the prototype detector arrays and use existing laser facilities to demonstrate the new capabilities.  
He or she will work with the LCLS Staff to secure time during early testing of the new soft x-ray undulators to demonstrate the improved window and resolution capabilities.

In addition to the budgeted RA, the PI has identified a Banting post-doctoral fellow, Audrey Therrien, who has needed expertise in charged particle simulations with GEANT4 as well as integrated circuit design and FPGA expertise.
The PI has also identified a Stanford graduate research fellow in Electrical Engineering, Matt Feldman, who holds expertise in digital signal processing in FPGA as well as deploying machine learning models (inference engines) to FPGA.

The PI will continue to leverage his group alumni fellows, most notably Wofram Helml (Univ.~Dortmund), Anton Lidahl (Quamcom Inc.), and Markus Ilchen (Euro XFEL/Univ.~Kassel) who, along with then graduate student Nick Hartmann (Coherent Inc.), formed the original attosecond angular streaking team within the PI's research group.
The PI expects these collaborative relationships to continue, having grown to include also a long standing collaboration with Andreas Galler who is currently organizing the EuroXFEL angular streaking effort as well as with Reinhard Keinberger of TU Munich/MPQ Garching Germany who is a familiar name in attosecond laser science.
Furthermore, the ongoing close relationship with Thomas Feurer of University of Bern Switzerland will ensure continued development of the attosecond angular streaking retrieval algorithms.

\paragraph{Expected publications}
\begin{itemize}
\item``Attoclock Ptychography'' -- Tobias Schweizer \textit{et al.} \cite{Feurer2018},
\item ``Online, single-shot characterization of few-femtosecond X-ray temporal pulse substructures at free-electron lasers via angular streaking'' -- Rupert Heider \textit{et al.},
\item ``Machine Learning enabled x-ray pulse reconstruction'' --  Gregor Hartmann \textit{et al.},
\item ``Machine Learning for streaming x-ray pulse characterization: spectrum, polarization, and time,''
\item ``Shaped attosecond x-ray FEL pulses for nonlinear x-ray science,''
\item ``Direct observation of laser-mixing of valence electronic symmetries.''
\end{itemize}
