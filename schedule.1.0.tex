\subsection*{Organization of Major Activities}\addcontentsline{toc}{subsection}{Organization of Major Activities}
%\begin{table}
%\caption{\label{tab::schedule} Timeline of the R\&D schedule. Project Years (PY) are sub-divided into 6 month blocks. }
\small
\begin{tabular}{l|l|l|l}
\hline
\multicolumn{2}{c|}{Project Year 1} &  \multicolumn{2}{c}{Project Year 2} \\
\hline
eTOF spectrometer design & Simulate final design & Test inital MCPs & Test Final system\\
Initial MCP electronics & Install initial MCPs & Complete MCP set & Final install\\
Chamber Design & Purchase Chamber& Install inital spectrometers & Complete Chamber\\
Initial Digitizers & Simulate waveform sampling& Complete Digitizer set & Minimize cross-talk\\
\hline
Purchase SystemOnChip & & Purchase FPGA &\\
Develop algorithms& Train ML on sims & Transfer Learning & Benchmark latency\\
\hline
\end{tabular}
%\end{table}
\normalsize


\begin{comment}
\paragraph*{Period 0.0: 7/1/2018 -- 12/31/2018\\}
Test 2$\mu$m streaking with attosecond pulses from LCLS-I using Axial VMI \cite{Siqi2018}.\\
\textbf{Publication(s): }

\paragraph*{Period 0.5: 1/1/2019 -- 6/31/2019\\}
Design detectors for array, use existing data to develop algorithms for streaming analysis.\\
Implement on-board FPGA processing and demonstrate/simulate calibration learning, detail implementation plan for LCLS-II.\\
\textbf{Publication(s): }
``Online, single-shot characterization of few-femtosecond X-ray temporal pulse substructures at free-electron lasers via angular streaking'' -- Rupert Heider \textit{et al.}
``Machine Learning enabled x-ray pulse reconstruction'' --  Gregor Hartmann \textit{et al.}

\paragraph*{Period 1.0: 7/1/2019 -- 12/31/2019\\}
Construct detector array prototype for Building 40 benchmarks and LCLS-II beneficial occupancy.\\
\textbf{Publication(s): }
``Benchmarking/Novel algorithm for real-time x-ray pulse characterization at the LCLS-II.''

\paragraph*{Period 1.5: 1/1/2020 -- 6/31/2020, instrument commissioning\\}
Install array on LCLS-II and benchmark resolution with XAFS using 3 $\mu$m laser dressing of N$_2$O.
\textbf{Publication(s): }
``Streaming attosecond resolution x-ray pulse characterization: spectrum, polarization, and time.''
``Shaped attosecond x-ray FEL pulses for nonlinear x-ray science''
``Direct observation of laser-mixing of valence electronic symmetries.''
\end{comment}

\subsection*{Responsibilities of key project personnel}\addcontentsline{toc}{subsection}{Responsibilities of key project personnel}
The PI will be responsible for leading the design effort to ensure a fundamentally integrated system, from chamber design and detector hardware to signal readout and waveform analysis.
He will also be principally concerned with the pulse reconstruction algorithm and data analysis pipeline.

Co-investigator Peter Walter is a Staff Scientist in LCLS Science Research and Development and will principally lead the detector simulations and construction.
Peter will also primarily lead the installation for early testing of the final concepts.
Co-investigator James Cryan is a Staff Scientist in LCLS Science Research and Development and PULSE and he will host much of the detector euv-based commissioning in his lab.
James will also continue to lead the VMI-based approach as well as many of the experimental simulations.
Both James and Peter will have regular interactions with the RA to help guide the work.

The Research Associate (RA) will spend 100\% of his or her time over the 2-year proposal period.  
In Year 1, he or she will use existing computational infrastructure to simulate the electron Time-of-Flight detector array, iterating the design in order to minimize the cross-talk between eTOF spectrometers in the array.  
Furthermore, he or she will identify and order the electron multiplier assemblies and digitizer electronics that work together to optimize energy resolution and spectral window.  
In Year 2, the RA will focus on the construction of the prototype detector arrays and use existing laser facilities to demonstrate the new capabilities.  
He or she will work with the LCLS Staff to secure time during early testing of the new soft x-ray undulators to demonstrate the improved window and resolution capabilities.
In addition to the budgeted RA, the PI has identified a Banting post-doctoral fellow, Audrey Therrien, who has needed expertise in charged particle simulations with GEANT4 as well as integrated circuit design and FPGA expertise.
The PI has also identified a Stanford graduate research fellow in Electrical Engineering, Matt Feldman, who holds expertise in digital signal processing in FPGA as well as deploying machine learning models to FPGA.

The PI will continue to leverage his group alumni fellows, most notably Wofram Helml (Marie Curie Foundation), Anton Lidahl (Wallenberg Foundation), and Markus Ilchen (Volkswagen Foundation) who, along with then graduate student Nick Hartmann, formed the original attosecond angular streaking team within the PI's research group.
The PI expects these collaborative relationships to continue, having grown to include also a long standing collaboration with Andreas Galler who is organizing the EuroXFEL angular streaking effort as well as with Reinhard Keinberger of TU Munich/MPQ Garching Germany who is a familiar name in attosecond laser science.
Furthermore, the ongoing close relationship with Thomas Feurer of University of Bern Switzerland will ensure continued development of the attosecond angular streaking retrieval algorithms.

\paragraph{Expected publications}
``Attoclock Ptychography'' -- Tobias Schweizer \textit{et al.} \cite{Feurer2018},
``Online, single-shot characterization of few-femtosecond X-ray temporal pulse substructures at free-electron lasers via angular streaking'' -- Rupert Heider \textit{et al.},
``Machine Learning enabled x-ray pulse reconstruction'' --  Gregor Hartmann \textit{et al.},
``Machine Learning for streaming x-ray pulse characterization: spectrum, polarization, and time,''
``Shaped attosecond x-ray FEL pulses for nonlinear x-ray science,''
``Direct observation of laser-mixing of valence electronic symmetries.''
