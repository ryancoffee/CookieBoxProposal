
There is increasing momentum in the development of spectro-temporally shaped x-ray FEL pulses \cite{eehg2009,Lutman13_twocolor,Marinelli13_twocolor,Allaria2014,Marinelli2015,Hemsing2016,Prince2016,Lutman2016,LutmanFreshSlice2016,Marinelli2016} in response to the rising tide of demand \cite{Cederbaum2008,Mukamel2007,Biggs2012,Mukamel2013,4WaveMixing,TIGER2015,ArtemFOA}.
The predominant method to characterize such novel temporal profiles is based on an x-band transverse accelerating cavity (XTCAV) \cite{xtcav2014} whereby the spent electron bunch is deflected horizontally, streaked in time by the phase of the transverse accelerating field.
A bending magnet then deflects this time-streaked beam vertically proportional to the energy.
Imaging the result, one records the time-energy distribution of the spent bunch.
This technique has been a critical tool for developing the recent x-ray FEL pulse shaping methods \cite{Marinelli2015,Marinelli2016,LutmanFreshSlice2016,Lutman2016}.
Unfortunately, it indirectly measures the x-ray temporal profile by identifying the imprint of lasing on the electron bunch.
Furthermore, barring a superconducting upgrade to the x-band cavity, the XTCAV can only run at 120 Hz, providing only intermittent information at best.

\begin{figure}[b]
%\begin{wrapfigure}[10]{r}{.65\linewidth}
\vspace{-\baselineskip}
\centerline{
	\includegraphics[width=\linewidth]{naturePhoton_NickWolfi.clockfig.eps}
}
\vspace{-\baselineskip}
\caption{
	\label{streakingschematic} 
	Schematic of angular photo-electron streaking. \cite{Nick2018}
}
%\end{wrapfigure}
\end{figure}

Since the inception of streaking as a pulse reconstruction method \cite{Zenghu2005,KellerAngularStreaking} there have been great recent gains in diagnosing laser-based pulses that encroach on the soft x-ray regime \cite{Biegert2016,WornerSci2017,Worner2017}.
Photo-electron streaking, a direct interaction, has the capability to measure the instantaneous temporal structure of x-ray pulses \cite{Hentschel2001}.
Some phase retrieval analysis require interferences with reference oscillators as in Refs.~\cite{Zenghu2010,Cocke2013}, we favor a more flexible scheme that can also accommodate \textit{in situ} x-ray experiments as well as the pulse retrieval diagnostic.
We will therefore focus on the angular array of electron spectrometers that are more compatible with the x-ray regime 0.2keV and up, as used for our recend proof of concept in Ref.~\cite{Nick2018}.
In x-ray photo-electron streaking, a noble gas like neon is dressed by a strong long wavelegnth infrared or THz field \cite{Helml2014,Juranic2014,Schulz2015}.
The streaking field vector potential shifts the outgoing photo-electron energy depending on the phase of the field at the time of photoionization.
In the more common linear polarized streaking \cite{Helml2014,Juranic2014,Schulz2015,Matthias2016}, the intensity profile of the photo-electrons versus their shifted energies can be mapped to the time-variation of the vector potential in a familiar ``streak camera'' interpretation, of course given that the x-ray pulse arrives near the zero-crossing of streaking vector potential.
There have been recent developments by the Cavalieri group to arrange two photo-electron spectrometers to sample the focal volume of the streaking field at two different places across the Gouy phase of the focus.
This in principle relieves the need for a zero-crossing carrier field, it is particularly sensative to the exact focussing conditions of the streaking laser.
However, in angular streaking, as depicted in Fig.~\ref{streakingschematic}, that vector potential is made to rotate by a using circular polarized long-wavelength field.
We thus intend to use the angular streaking laser field as a ``clock'' that imprints time into the electron spectra and thus allows for a time-resolved interpretation of electron emission \cite{KellerNonadiabatic2013}


We will leverage our long history with photo-electron streaking at the LCLS \cite{Duesterer11,Meyer12,Helml2014} by extending the attosecond angular streaking method of Refs.~\cite{CorkumAngularStreaking,KellerAngularStreaking} to the x-ray regime.
Pulse-to-pulse variations at an FEL require single-shot measurements much like the velocity map imaging (VMI) \cite{VrakkingRSI} extension of attosecond angular streaking \cite{attoclockVMI2013}.
Although we have explored such a single-shot VMI solution for 120Hz operation \cite{Siqi2018}, the requirement of a two-dimensional detector precludes high repetition rates. 
Furthermore, the single resolution window of the VMI precludes its use for the kinds of novel multi-color pulses we have come to expect from LCLS.

\begin{wrapfigure}[22]{l}{3.5in}
\vspace{-0.5\baselineskip}
\centerline{\includegraphics[width=\linewidth]{naturePhoton_NickWolfi.spectrograms.eps}}
\vspace{-0.5\baselineskip}
\caption{\label{fig::retrievals}X-ray pulse shape retrievals reproduced from Ref.~\cite{Nick2018}.}
\end{wrapfigure}
As an alternative, we repurposed what was originally considered for x-ray FEL polarimeter \cite{Markus2014,Allaria2014,Mazza2014,Lutman2016} to measure the angularly streaked photo-electron spectra with an angular array of 16 electron Time-of-Flight (eTOF) detectors as depicted in Fig.~\ref{streakingschematic} \cite{Nick2018}.
The result was an x-ray pulse temporal reconstruction with 500 attoseconds resolution owing to the poor energy resolution and conservatively long 10~$\mu$m wavelength for the streaking field. 
Given that the array of eTOFs was originally used at LCLS as a polarization diagnostic, we also target its use for a full spectral and polarization reconstruction with better than 250 attosecond temporal resolution and 0.25 eV energy resolution.

Normally, the x-ray pulse produces undressed neon photo-electrons that distribute into a dipole probability distribution for linearly polarized x-rays, and a circular pattern for random or circular x-ray polarization with a common kinetic energy regardless of emission angle.
When dressed with the circularly polarized laser field (Fig.~\ref{streakingschematic}) those electrons receive a momentum kick toward the instantaneous direction of the vector potential in a reference frame that spirals relative to the lab frame at the carrier cycle frequency.
In this way, one detector will measure electrons with an excess of energy, the opposite detector with less energy, and the two orthogonal detectors will measure the photo-electrons as the projected vector-potential sweeps through a zero-crossing.
The additional detectors further constrain the pulse shape retrieval shown in Fig.~\ref{fig::retrievals}.

