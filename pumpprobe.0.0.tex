Many attosecond scale experiments are currently only enabled by high-harmonic generation (HHG)\cite{Lewenstein1994,Hentschel2001,Chen2014,Biegert2014,Schmidt2016,Biegert2016,WornerSci2017,Zenghu2017}.
These experiments could greatly benefit from the much higher brightness of an attosecond xFEL beamline \cite{Ding2009,Xiang2009,xLEAP}.
They would effectively exchange the flux challenge of the HHG sources for the synchronization challenge of xFEL pulses.
With the ability for x-ray pulse characterization at the attosecond level, one removes the synchronization challenge clearly for x-ray pump/x-ray probe experiments.
Figure~\ref{fig::doublepulses} indeed shows that angular streaking can not only identify double pulses, but also sort such pulses into relative delay \cite{Nick2018}.
\begin{wrapfigure}[21]{l}{.5\linewidth}
\centerline{\includegraphics[width=.8\linewidth]{naturePhoton_NickWolfi.doublepulses.eps}}
\caption{\label{fig::doublepulses}Reproduced from Ref.~\cite{Nick2018}.
Roughly 1\% of the SASE pulses show only two spikes when running in low charge mode with emittance shaping.
Such pulses can then be sorted, allowing a ``measure-and-sort'' x-ray pump/x-ray probe experimental paradigm.}
\end{wrapfigure}

In fact, if one prefers a weaker isolated attosecond pulse for a probe pulse, one could use a more traditional laser-based HHG source.
Such pulses in the euv could be used to pump inner valence transitions in molecular systems, setting up rather pronounced coherent electronic motions, or simply used as broad-band transient absorption probes \cite{Biegert2016}.
Then the attosecond x-ray pulses from the LCLS could be used to time resolve the valence occupation via time-resolved photo and Auger electron spectroscopy.
In this case, the euv light is sufficient to produce attosecond bursts of electrons from a helium buffer gas while the x-ray pulses would equally well produce high energy photoelectrons also from helium.
In this way, angular streaking could simultaneously recover, on a single shot basis, the relative delay between x-ray/x-ray and even euv/x-ray attosecond pulse pairs.

