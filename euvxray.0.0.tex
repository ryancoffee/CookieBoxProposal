The primary obstacle that detracts the attosecond community from using an xFEL source is the lack of intrinsic synchronization enjoyed in lower-fluence lab-based HHG sources.


Many attosecond scale experiments, currently only enabled by high-harmonic generation (HHG)\cite{Lewenstein1994,Hentschel2001,Chen2014,Schmidt2016,WornerSci2017}, would greatly benefit from the much higher brightness of an attosecond xFEL beamline \cite{Ding2009,Xiang2009}.
This would effectively exchange the flux challenge of HHG sources for the challenge of synchronizing optical and xFEL pulses with 100 attosecond precision.
Helium buffer gas with 20eV photoelectrons into grounded detectors will enable angular streaking simultaneously of EUV attosecond pulses with x-ray attosecond pulses in the identical scheme as the x-ray atto pairs.
This way one could use the weaker laser based isolated attosecond pulses \cite{Biegert2016} as a supercontinuum probe up to the carbon K-edge and use the LCLS-II attosecond pulses for the much stronger $1s\rightarrow\mbox{valence}$ resonant pump transitions that would trigger the valence electronic correlations from a chosen atomic site in the molecule.

Above 50 eV attosecond pulses we will be able to directly measure both x-ray and EUV attosecond pulses with the cookiebox.

Attosecond level synchronization is important to LCLS-II.
In this case, one can imagine using the plateau region of typical laser-based HHG sources in the 30-60 eV regime to form EUV isolated attosecond pulses.
Such pulses can be used to pump inner valence transitions in molecular systems, setting up rather pronounced coherent electronic motions.
Then attosecond x-ray pulses from the LCLS can be used to time resolve the valence occupation via NEXAFS and xray transient absorption spectroscopy.
In this case, the euv light is sufficient to produce attosecond bursts of electrons from a helium buffer gas while the x-ray pulses would equally well produce high energy photoelectrons also from helium.
We therefore plan for a detector that will simultaneously recover, on a single shot basis, the relative delay between the euv and the x-ray attosecond pulses.
This also suggests a very far detuned window of energy acceptance for the spectrometers that make up the angular array.

One can easily imagine using the soft x-ray pulses form LCLS as a strong resonant $1s\rightarrow$valence pump wiht a weak transient absorption probe in the euv regime \cite{Biegert2014,Biegert2016,WornerSci2017}.
