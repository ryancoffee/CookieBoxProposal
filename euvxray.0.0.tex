Above 50 eV attosecond pulses we will be able to directly measure both x-ray and EUV attosecond pulses with the cookiebox.

Attosecond level synchronization is important to LCLS-II.
In this case, one can imagine using the plateau region of typical laser-based HHG sources in the 30-60 eV regime to form EUV isolated attosecond pulses.
Such pulses can be used to pump inner valence transitions in molecular systems, setting up rather pronounced coherent electronic motions.
Then attosecond x-ray pulses from the LCLS can be used to time resolve the valence occupation via NEXAFS and xray transient absorption spectroscopy.
In this case, the euv light is sufficient to produce attosecond bursts of electrons from a helium buffer gas while the x-ray pulses would equally well produce high energy photoelectrons also from helium.
We therefore plan for a detector that will simultaneously recover, on a single shot basis, the relative delay between the euv and the x-ray attosecond pulses.
This also suggests a very far detuned window of energy acceptance for the spectrometers that make up the angular array.

One can easily imagine using the soft x-ray pulses form LCLS as a strong resonant $1s\rightarrow$valence pump wiht a weak transient absorption probe in the euv regime \cite{Biegert2014,Biegert2016,WornerSci2017}.
