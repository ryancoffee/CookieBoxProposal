
\appendix
\section*{Appendix 2: Current and Pending Support}\addcontentsline{toc}{section}{Appendix 2: Current and Pending Support}
Both current and pending support will be predominantly covered under the U.S. Department of Energy / Stanford University Contract for Management and Operation of SLAC National Accelerator Laboratory with a small fraction under the National Institute of Health.\\
\vspace{\baselineskip}\\
\begin{tabular}{lll}
Current Support & LCLS-Soft X-ray Department & 50\%  \\
		& LCLS High Sensitivity Timing & 20\% \\
		& LDRD Machine Learning for LCLS-II & 20\%  \\
		& NIH Time-of-Flight PET & 10\% \\
		\vspace{2\baselineskip} &&\\
Pending support&same as current & \\
\end{tabular}

\clearpage
\appendix
\section*{Appendix 3: Bibliography and References Cited}\addcontentsline{toc}{section}{Appendix 3: Bibliography and References Cited}

\bibliographystyle{unsrt}
\bibliography{$BIBFILES/hhg,$BIBFILES/medicine,$BIBFILES/mariano,$BIBFILES/collaborations,$BIBFILES/polarization,$BIBFILES/vmi,$BIBFILES/timing,$BIBFILES/carriers,$BIBFILES/rixs,$BIBFILES/ued,$BIBFILES/aromatic,$BIBFILES/lcls_refs,$BIBFILES/shaping,$BIBFILES/streaking,$BIBFILES/supercontinuum,$BIBFILES/fel_multicolor,$BIBFILES/attosecond,$BIBFILES/Coffee,$BIBFILES/xrayspectroscopy,$BIBFILES/computing,$BIBFILES/cookiebox,$BIBFILES/Bucksbaum,$BIBFILES/time-frequency,$BIBFILES/deepuv,$BIBFILES/Gibson,$BIBFILES/materials}

\clearpage
\appendix
\section*{Appendix 4: Facilities and Other Resources}\addcontentsline{toc}{section}{Appendix 4: Facilities and Other Resources}

There are two principle facilities identified for this project.
\begin{itemize}
\item Early commissioning and development experiments will be carried out in the Photon Science Laboratory Building (PSLB).
\item \textit{in situ} testing and experiments will be carried in Hutch 1.1 of the Near Experimental Hallout in the Photon Science Laboratory Building (PSLB).
\end{itemize}

Analysis resources exist from both the SLAC-unix farm and the LCLS-unix farms.  
Our long collaboration with the data analysis and controls groups at LCLS not only allows the PI particularly early insight into the computing resources, but is also motivates his active pursuit of data compression and on-board analysis algorithms.
This mutual benefit ensures the continued use and support for these computing facilities.

Office space will be available for team members with locations divided into available space in the building 901 LCLS office building with additional space for the Coffee, the RA, students and visiting scientists in the PULSE Institute of Building 40.

\subsection*{Additional Personnel}%\addcontentsline{toc}{subsection}{Additional Personnel}

We expect one graduate student, recruited from Stanford University, who will work on this project.
This student will participate via the student outreach programs of LCLS at SLAC.

\clearpage
\appendix
\section*{Appendix 5: Equipment}\addcontentsline{toc}{section}{Appendix 5: Equipment}

Required existing equipment includes:
\vspace{\baselineskip}
with associated laser systems: 
\begin{itemize}
\item We will continue to work in collaboration with James Cryan for intermittent testing of the spectrometer array in the EUV PULSE Lab in Building 40 at SLAC.
\item The R\&D laser lab in the PSLB will be used for detector testing and construction as the space comes available in Fall 2018.
\end{itemize}
\vspace{\baselineskip}

\noindent Other equipment available to the project include:
Laser conversion for generation of 2 $\mu$m pulses as exists for LCLS-I.

\subsection*{Materials \& Supplies}
year 1
\begin{enumerate}
\item Vacuum Chamber 
\item 2x MCP electronics (15.8)
\item 1x 22k Digitizer + PCIe carrier, = 2x6GSps 
\item 4U computer (6K)
\item Xilinx Zynq UltraScale+ RFSoC ZCU111 Evaluation Kit (9)
\end{enumerate}

year 2
\begin{enumerate}
\item AIR-TEC electronics technician 120Hrs
\item 2x 22k Digitizer+PCIe carrier, gives a total of (3 digitizers total, each running 2x6GSps for total of 6 channels)
\item 4x MCP electronics (31.5)
\item FPGA for streaming analysis Virtex at 7k  
\item Miscellaneous vacuum and electronics M\&S (14k)
\end{enumerate}

\clearpage
\appendix
\section*{Appendix 6: Data Management Plan}\addcontentsline{toc}{section}{Appendix 6: Data Management Plan}

As stated in the project narrative, one of the central themes is to reduce the data load.
The data that is accumulated as part of this project will be made broadly available both internally via SLAC/LCLS unix user account access and, by inquiry to the PI, externally via coordinated data formatting and access FTP.

Data taken with the LCLS will be stored in accordance with LCLS policy also in SLAC Central Storage (or LCLS storage if in future LCLS moves the storage service).
Access can then be granted by the PI also for any individual who obtains an LCLS user unix account. 

The laser lab based data will be housed in the SLAC Central Computing.  
The PI currently maintains a 1 TB/year subscription and that will be incrementally increased up to a 5 TB/year storage, expanded when needed.
%The subscription cost is included in the budget.
The data will be made broadly available by SLAC unix account access and externally by contacting the PI and coordinating an FTP service of the data.
