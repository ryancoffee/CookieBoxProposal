The scope of this project is the Research and Development required for the design of an electron spectrometer system that is capable of reconstructing the x-ray pulse shapes with attosecond time resolution, 0.25~eV spectral resolution, and sub-pulse polarization resolution.
It should function as a single-shot diagnostic for multi-pulse xFEL configurations that span hundreds of eV photon energy difference.
Motivated by the recent experimental success of Ref.~\cite{Nick2018} and further by the pulse reconstruction efforts of Refs.~\cite{Siqi2018,Feurer2018}, we target an angular array of newly designed electron Time-of-Flight spectrometers.  
The project will deliver a design that meets specifications for each eTOF spectrometer, including mutual interaction considerations in the angular array.
These specifications are set by the needs for reliable reconstruction of both SASE and x-ray pulse pairs of complicated polarization, multi-color spectrum, and temporal structure; more succinctly, the polarization resolving attosecond Time-Energy distribution.

The project will also deliver an integrated data pipeline plan.
We will use the simulations that result from the designing the optimised detectors to also develop in parallel the optimized analysis stream.  
From simulations we will benchmark the retrieval algorithms of Refs.~\cite{Nick2018,Siqi2018,Feurer2018} against the new specifications of higher energy resolution over wider energy windows.
We will match not only the electronics to the expected signals, but also the on-board pre-analysis in the overall data pipeline.
Since we target at least 100~kfps analysis throughput, ideally with microsecond level latency, we expect the need to use high-speed machine learning inference.
Therefore we will integrate the inferencing with the early stages of the digitization and FPGA-based signal processing.
The output of the inference engine is expected to inform user detectors how to route buffered data, either into sorted order for histogram updates or for intelligent veto.

