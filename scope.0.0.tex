The scope of this project is the Research and Development needed to produce an angular array of 16 novel electron Time-of-Flight spectrometers.  
The specifications for this angular array are set by the needs for the reliable reconstruction of both SASE x-ray pulses and x-ray pulse pairs of complicated multi-color spectrum, polarization, and temporal structure; more succinctly, the polarization resolving Time-Energy distribution.
The aim for the new design is to achieve the 0.25\% resolution while retaining the ability for strong retardation fields without significantly sacrificing the collection efficiency.  



In addition to the detector design in this preparation phase, we will use existing data as well as simulations to build a repository of example waveforms with realistic expectations for the point-spread functions.  
We will use this simulated data to test both the time-domain ptychography method of pulse reconstruction as well as a machine learning method for fast predictive inference.
By performing both algorithms on the same simulations, we can benchmark the pre-analysis that will ultimately be used to enhance resolution in the retrieval while maintaining the very high throughput needed for real-time analysis.

In the beginning of Phase 1, the two prototype detectors will be ordered along with the final design vacuum chamber. 
The bench testing of the two prototype detectors will test for the detector cross-talk—a major impediment with the design used in Ref. ii—and for detector ringing, e.g. optimizing that point-spread function.  
The detectors will be installed into the final design vacuum chamber since the detectors mate directly into the chamber.  
Since fringe electric fields may compromise the final electron energy resolution, along with stray magnetic fields, we must test the prototype detector pair for cross-talk and ringing in the final chamber environment.  
After confirming the specifications of the protoype detectors, and any modifications required, we will then finalize the angular array with the remaining 14 eToF modules.

The signal testing in Phase 1 will also provide early insight into the actual signal processing that will be needed in the final build out.  
Therefore, we will use this to develop in parallel the optimized analysis stream.  
This chain of analysis—be it ptychography or a machine learning inference model or, more likely, an interwoven fusion of the two—will be compiled for FPGA deployment.  
By deploying to FPGA, we will only have need to transfer the reconstructed x-ray pulse description rather than each of the individual raw waveforms. 
We note that preliminary results indicate that, although the ptychographic reconstruction is computationally expensive, a preliminary artificial neural network inference model has been trained on simulations and performs with micro-second latency.  
Based on our close collaboration with the Stanford CS/EE group of Kunle Olukotun, we expect such a solution could be deployed to FPGA and run in streaming inference mode.

In the beginning of Phase 2, we will complete the detector array and begin demonstration of its capability when the LCLS comes back.  
We will use the 120Hz machine running into the soft x-ray undulators to demonstrate the spectral range of the detectors and the resolution.  
We will also develop the calibration protocol which will then be automated as the system becomes integrated into the LCLS-II controls system and DAQ.  
We note that expected configuration of an ML-enabled data reduction node is currently being designed with this detector as one of the use-cases for on-board streaming analysis.  
We will use Phase 2 to work directly with the group of Jana Thayer to ensure full compatibility with the LCLS-II Data Reduction Pipeline (DRP).  
By Summer of 2020, we expect the system would be ready for full deployment, just in time to receive LCLS-II, high repetition rate pulses.

