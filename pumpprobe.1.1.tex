
Many attosecond scale experiments that are currently only enabled by high-harmonic generation (HHG)\cite{Lewenstein1994,Hentschel2001,attoclock2008,Chen2014,Biegert2014,Schmidt2016,Biegert2016,WornerSci2017,Zenghu2017} could greatly benefit from the much higher brightness of an attosecond xFEL beamline \cite{Ding2009,Xiang2009,xLEAP}.
They would effectively exchange their current flux challenge for the synchronization challenge of xFEL pulses.
However, the ability for x-ray pulse characterization at the attosecond level removes the synchronization challenge for x-ray pump/x-ray probe experiments.
Figure~\ref{fig::doublepulses} indeed shows that angular streaking can not only identify double pulses, but also sort such pulses into relative delay \cite{Nick2018}.
In that demonstration roughly 1\% of the SASE pulses measured consisted of only two spikes when running in low charge mode with emittance shaping \cite{EmmaFoil,Ding2015}.
Such pulses can then be sorted, allowing a familiar ``measure-and-sort'' x-ray pump/x-ray probe experimental paradigm.

\begin{wrapfigure}[12]{r}{3in}
\centerline{\includegraphics[width=\linewidth]{naturePhoton_NickWolfi.doublepulses.eps}}
\vspace{-0.5\baselineskip}
\caption{\label{fig::doublepulses}Reproduced from Ref.~\cite{Nick2018}.
}
\end{wrapfigure}

At the FEL, we are free to vary the x-ray photon energy of the pulse pairs independently \cite{Lutman13_twocolor,LutmanFreshSlice2016} and also their polarization states \cite{Lutman2016,Dichroism2016}.
This is a key feature for studying the kind of chirality dynamics that was listed as a major scientific thrust area for LCLS-II \cite{lcls2_opportunities}.
One could imagine using one color of x-ray pulse to pump a chiral molecule such as trifluoromethyloxirane \cite{Ilchen2017} at a fluorine or cabon atom and then probe the evolution of the valence electronic response via XAFS at the oxygen edge.
This requires different polarizations \cite{Lutman2016,Dichroism2016} where also one pulse is tuned to the oxygen edge and the other is tuned to the carbon or fluorine edge.
Inspired by Fig.~\ref{fig::2color2polScheme}, such a capability is enabled by the combination of Delta undulator for polarization control together with the variable gap soft x-ray undulator of LCLS-II in split mode \cite{Lutman13_twocolor,LutmanFreshSlice2016}.
Captializing on this novel x-ray shaping mode sets the requirement that our attosecond pulse diagnostic allow for also polarization determination for pulse pairs separated by well over to 200~eV.

\begin{figure}[b]
\centerline{\includegraphics[width=\linewidth]{nphoton.2016.79.CookieBoxScheme.eps}}
\caption{\label{fig::2color2polScheme} Delta and split undulator scheme for multi-color, multi-polarization, x-ray pulse pair generation reproduced from Ref.~\cite{Lutman2016}. }
\end{figure}

We also note that, if instead, one prefers a weaker isolated attosecond probe pulse \cite{Chen2014,Schmidt2016,Biegert2016,WornerSci2017}, one could use a more traditional laser-based HHG source and still address the HHG/xFEL synchronization challenge.
Here, the euv light is sufficient to produce attosecond bursts of electrons from a helium buffer gas while the x-ray pulses would equally well produce high energy photoelectrons also from helium.
In this way, angular streaking could simultaneously recover, on a single shot basis, the relative delay between x-ray/x-ray and even euv/x-ray attosecond pulse pairs.
Attosocond pulses generated by the high-harmonic plateau region \cite{attoclock2008,Kulander2011} could be used to induce inner valence transitions in molecular systems, setting up rather pronounced coherent electronic motions.
Then the attosecond x-ray pulses from the LCLS could be used to interrogate the valence occupation via time-resolved photo and Auger electron spectroscopy.
Alternatively, one could use the weaker HHG isolated attosecond pulses \cite{Biegert2016} as a supercontinuum probe up to the carbon K-edge.
The LCLS-II would provide the much higher power attosecond pump pulses for $1s\rightarrow\mbox{valence}$ resonant transitions.
This would allow for strong pumping of valence electronic correlations from a chosen atomic site in the molecule well above nitrogen and oxygen, even fluorine and transition metal $L$-edges.

