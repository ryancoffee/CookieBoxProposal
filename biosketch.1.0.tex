\subsection*{Ryan Coffee}
LCLS Laser Department\hfill Tel:650.387.0981\\
SLAC National Accelerator Laboratory\hfill Fax:650.926.2521\\
Mail Stop 20, Menlo Park, California  94025\hfill E-mail: coffee@slac.stanford.edu\\

\subsection*{Education and Training}
%\addcontentsline{toc}{subsection}{Education and Training}
\begin{tabular}{llll}
Research Associate & & SLAC National Accelerator Laboratory & 06/2006--04/2009\\
Ph.D. & Physics & University of Connecticut & 06/2006\\
M.S. & Physics & University of Connecticut & 12/2001\\
B.S & Physics & University of Arkansas & 06/1999\\
B.A. & Philosophy & University of Arkansas & 06/1999
\end{tabular}

\subsection*{Research and Professional Experience}
%\addcontentsline{toc}{subsection}{Research and Professional Experience}
\textbf{01/2014--present Staff Scientist}, PULSE Institute\\
\textbf{04/2009--present Staff Scientist}, LCLS Laser Division, SLAC\\
Spectral and spectrogram encoding of relative x-ray arrival time, 
sub-10 fs pulse generation for FEL multiplicative seeding and for time resolved photo-chemistry, 
optical and THz laser streaking techniques at the LCLS, 
angle-resolved double- and single-core hole spectroscopy of impulsively-aligned molecules, 
x-ray pump/x-ray probe experiments at LCLS, 
x-ray pulse shaping for multi-dimensional x-ray spectroscopy, 
gas phase ultrafast electron diffraction, 
LCLS experimental laser facility installation and commissioning\\
\textbf{06/2006--04/2009 Research Associate}, PULSE Institute\\
Coherent control of rotational wave-packet motion in ambient nitrogen and iodine.\\
\textbf{01/2006--06/2006 Research Associate}, University of Michigan\\
Participation in two of the final SPPS experiments\\
\textbf{09/1999--06/2006 Research Assistant}, Department of Physics, University of Connecticut\\
Two-color pump-probe optical experiments with nitrogen, 
molecular vibrational wave-packet motion on laser induced potential energy surfaces, 
ion time-of-flight spectroscopy, 
vuv-fluorescence spectroscopy of selective high-order multi-photon absorption in N$_2$, 
transient absorption spectroscopy.

\setlist{nosep}%,after=\vspace{\baselineskip}}

\subsection*{Selected publications}
%\addcontentsline{toc}{subsection}{Selected publications}
\begin{enumerate}
\item \textit{Attoclock Ptychography}
T Schweizer, \etal \textbf{R Coffee}, and T Feurer 
Applied Sciences, \textbf{8} 1039 (2018)
\item \textit{Attosecond time–energy structure of X-ray free-electron laser pulses}
N Hartmann, \etal \textbf{R Coffee}, and W. Helml
Nature Photonics, \textbf{12} 215 (2018)
\item \textit{Characterizing isolated attosecond pulses with
angular streaking}
S Li, \ldots \textbf{R Coffee}\etal 
Optics Express, \textbf{26} 4531 (2018)
\item \textit{Optical Shaping of X-Ray Free-Electron Lasers}
A Marinelli, \textbf{R Coffee}\etal
Physical Review Letters, \textbf{116}, 254801 (2016)
\item \textit{Polarization control in an X-ray free-electron laser}
AA Lutman \ldots \textbf{R Coffee}\etal 
Nature Photonics, \textbf{10}, 468 (2016)
\item \textit{Generating femtosecond X-ray pulses using an emittance-spoiling foil in free-electron lasers}
Y Ding, C Behrens, \textbf{R Coffee}\etal
Applied Physics Letters \textbf{107}, 191104 (2015)
\item \textit{High-intensity double-pulse X-ray free-electron laser}
A Marinelli, \ldots \textbf{R Coffee}\etal
Nature Communications \textbf{6}, 6369 (2015)
\item \textit{Measuring the temporal structure of few-femtosecond FEL X-ray pulses directly in the time domain}
W Helml, \ldots \textbf{R Coffee}\etal
Nature Photonics, \textbf{8}, 950 (2014)
\item \textit{Sub-femtosecond precision measurement of relative X-ray arrival time for free-electron lasers}
N Hartmann,\etal \textbf{RN Coffee}
Nature Photonics \textbf{8}, 706 (2014)
\item \textit{Spectral encoding method for measuring the relative arrival time between x-ray/optical pulses}
M Bionta\etal \textbf{R Coffee}
Review of Scientific Instruments, \textbf{85}, 083116 (2014)
\item \textit{Multicolor Operation and Spectral Control in a Gain-Modulated X-Ray Free-Electron Laser}
A Marinelli,\etal\ldots \textbf{RN Coffee}, and C Pellegrini
Physical Review Letters \textbf{111}, 134801 (2013)
\item \textit{Experimental demonstration of femtosecond two-color x-ray free-electron lasers}
AA Lutman, R Coffee\etal
Physical Review Letters \textbf{110}, 134801 (2013)
\item \textit{Spectral encoding of x-ray/optical relative delay}
Mina R. Bionta,\etal \textbf{R. N. Coffee}
Optics Express, \textbf{19}, 21855 (2011)
\end{enumerate}


\subsection*{Synergistic Activities}
%\addcontentsline{toc}{subsection}{Synergistic Activities}

\paragraph*{Ultrafast electron diffraction (UED)}
%\addcontentsline{toc}{subsubsection}{Ultrafast electron diffraction (UED)}
The PI has recently become intimately involved with the use of ultrafast electron diffraction (UED) in order to merge spectroscopic studies at the LCLS with the structural sensitivity of electron diffraction \cite{ued_rsi2015,Jie2016}.
The merger of the two experimental paradigms is enabled by induced time-domain coherent molecular motions that in-turn produce concerted fluctuations in both x-ray spectra and in electron diffraction features.
By pattern recognizing these common-mode fluctuations, one can merge experimental results as if the two experiments were performed simultaneously.
The PI is therefore keenly attuned to need for exquisite synchronization at high energy ultrafast electron diffraction sources.
The successful outcome of Objective~\ref{obj::controlling} would be immediately applicable to high energy electron diffraction facilities.% such as SLAC's UED facility, LBNL's facility, and others.


\paragraph*{Deep ultra-violet}
During period two, the LCLS facility will be down.
Here we will draw on our existing research program investigating routes to broad-band tunable deep uv.
This collaborative work together with the Keinberger group of TU Munich produced a Marie-Curie Fellowship project of Wolfram Helml to develop ultra-broadband, sub-10 fs duration deep ultraviolet pulses.
This project later led to a Masters student research project of Patrick Rupprecht to develop the required pulse characterization in the deep uv regime.
This deep ultraviolet will be used as a surrogate for the x-ray pulse during the LCLS down time from August 2018 -- August 2019.
This existing setup will continue to serve as a surrogate source for testing and debugging new concepts and materials. 

\paragraph*{Stanford Medical}
%\addcontentsline{toc}{subsubsection}{Stanford Medical}
The PI has recently opened a new collaboration with the group of Dr.~Craig Levin in the Division of Nuclear Medicine at Stanford.
Through this collaboration we expect a mutually beneficial investigation into novel scintillation materials for improving the cascade time and the signal per dose ratio.
Such a time resolution would greatly improve the spatial resolution of the PET scan per unit time spent undergoing the scan. 
Ideally one could imagine then making such scans in short time-series if each scan only takes a few seconds.
Then one can observe the dynamic processes in e.g.~brain activity.

\paragraph*{X-ray pulse shaping}
%\addcontentsline{toc}{subsubsection}{X-ray pulse shaping}
The PI has played a central role in motivating and helping the development of the many electron bunch based methods for x-ray pulse shaping \cite{Lutman13_twocolor,Marinelli13_twocolor,Helml2014,Marinelli2015,Lutman2016,Marinelli2016}.
From performing the first double-slotted foil experiment for two x-ray pulses \cite{CoffeeDAMOP11} to demonstrating optical carving of the electron bunch \cite{Marinelli2016} and helping demonstrate multi-polarization multi-color operation \cite{Lutman2016}, our group has helped to pioneer many of the pulse shaping schemes that have been developed thus far at the LCLS.
On the active forefront of pulse shaping, he is uniquely positioned to develop the pulse shape diagnostics hand-in-hand with the accelerator R\&D for which he is already an active contributor.
This close overlap with the LCLS Accelerator R\&D group is in fact what allows for the shared resources such as the two Project Scientist/RA positions to be shared so intimately.
These positions will be fundamentally interwoven between the development of novel FEL methods and the attosecond diagnostics required to interrogate those methods.


\paragraph*{Machine learning}
%\addcontentsline{toc}{subsubsection}{Machine learning}

Such so-called ``deep learning'' is used to interrogate the output of the artificial neural networks to check the physical interpretation of how the output is determined \cite{Mihir}.
Our preliminary test, shown in Fig.~\ref{mlsorted}, uses the neural network classifier to sort LCLS shots based on a CNN predicted two-color spectral distribution.
The sorted shots are shown at right and an example of an XTCAV image is shown in the top left of Fig.~\ref{mlsorted}.
Guided back propagation is then used to interrogate the neural network about which regions of that image influenced the sorting most strongly, and this is shown in the lower left.
These two regions indeed indicate the loss of electron energy due to the lasing process, thus boosting our confidence that the CNN is sorting based on valid physical properties of the x-ray pulse, photon energy separation in this case.

\subsection*{List of collaborators and co-authors: (48 months)}

\vspace{1\baselineskip}
\noindent\textbf{Collaborators:}\\
\vspace{-1.5\baselineskip}
{\small
\begin{multicols}{2}
\noindent 
	Lorenzo Avaldi \hfill CNR-ISM, Rome\\ 
	Nora Berrah \hfill Univ. of Connecticut\\
	Martin Beye \hfill HZB Berlin\\
	Christoph Bostedt \hfill ANL\\
	Marco Cammarata \hfill Univ. of Rennes, France\\
	Adrian Cavalieri \hfill CFEL Hamburg\\
	Martin Centurion \hfill U. Nebraska,\\
	Tilo Doeppner \hfill LLNL\\
	Gilles Doumy \hfill ANL \\
	Stefan D\"usterer \hfill FLASH DESY Hamburg\\
	Raimund Feifel \hfill Univ. of Gothenburg, Sweden\\
	Thomas Fennel \hfill Univ. Rostock, Germany
	Thomas Feurer \hfill Univ. of Bern, Switzerland\\
	Thornton Glover \hfill Gordon \& Betty Moore Found.\\
	Jan Gr\"unert \hfill Euro. XFEL, Hamburg\\
	Markus G\"uhr \hfill Potsdam University, Germany\\
	Marion Harmand \hfill IMPMC-UPMC, Paris, France\\
	Janos Hajdu \hfill Uppsala Univ. Sweden\\
	Christoph Hauri \hfill SwissFEL PSI, Switzerland\\
	Dan Kane \hfill Mesa Photonics, Albequerque\\
	Reinhard Kienberger \hfill TU Munich\\
	Jochen K\"upper \hfill CFEL, Hamburg\\
	Jerry LaRue \hfill Chapman, Irvine CA\\
	Jon Marangos \hfill Imperial College, UK\\
	Marc Messerschmidt \hfill BioXFEL, Hamburg\\
	Michael Meyer \hfill Euro. XFEL, Hamburg\\
	Catalin Miron \hfill ELI-Delivery Consortium\\
	Thomas M\"oller \hfill TU Berlin\\
	Serguei Molodtsov \hfill Euro XFEL, Hamburg\\
	Anders Nilsson \hfill Uppsala University, Sweden\\
	Steve Pratt \hfill ANL\\
	Artem Rudenko \hfill Kansas State University\\
	Daniel Rolles \hfill Kansas State University\\
	Nina Rohringer \hfill U. of Hamburg, Germany\\
	Arnaud Rouzee \hfill MBI Berlin\\
	Ilme Schlichting \hfill MPI Heidelberg\\
	Sharon Shwartz \hfill Bar-Ilan University, Israel\\
	Klaus Sokolowski-Tinten \hfill U. of Duisburg-Essen,\\\mbox{ } \hfill Essen Germany\\
	Thomas Tschentscher \hfill Euro. XFEL Hamburg\\
	Kiyoshi Ueda \hfill Tohoku Univ., Japan\\
	Joachim Ullrich \hfill PTB Germany\\
	Jens Viefhaus \hfill DESY
\end{multicols}
}
\normalsize
\subsection*{Graduate and Postdoctoral Advisors}
G.~Gibson (University of Connecticut), P.H.~Bucksbaum (PULSE/Stanford)





