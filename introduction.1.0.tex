\begin{wraptable}[14]{r}{.5\linewidth}
\vspace{-1.5\baselineskip}
\caption{Soft x-ray conditions for LCLS-I and the high-repetition rate LCLS-II. \cite{lcls2_opportunities}}\label{lcls2specs}
\begin{tabular}{lcr}
\toprule
Parameter & LCLS-I &LCLS-II\\
\midrule
Max rep.~rate & 120 Hz & 930 kHz\\
Average power & 0.5 W & 200--900 W\\ 
Pulse energy & 4 mJ & 0.1--5\footnotemark[1] mJ\\
\shortstack{Photon energy} & 0.25--2 keV & 0.2--5 keV\\
\shortstack{Arrival stability} & 100 fs rms& 20 fs rms\\
\toprule
\end{tabular}\\
\footnotemark[1] $\geq$ 200 $\mu$J typically at reduced repetition rates.  
Lower charge modes for short pulse operation can conserve peak power while allowing full repetition rate.
\end{wraptable}

The advent of x-ray free electron lasers (xFELs) has brought the ability to resolve ultrafast processes in molecular and material systems on their vibrational time and length scales \cite{Fritz2007,Katayama2013,Mariano2013,McFarland2014}.
The next horizon is upon us: we will use attosecond x-ray pulses to control and interrogate the correlated electronic motion as we delve into the age of the molecular electronic movie.
Given the importance of understanding such electronic flow in photo-excited systems (Fig.~\ref{fig::cederbaum}), there is a strong desire to drive electrons into concerted coherent motion \cite{Cederbaum2008,Biggs2012,Mukamel2013} and then probe the local electronic environment with time-resolved x-ray sepctroscopies.
This has even led to direct funding of a major research effort focused specifically on attacking this regime with FEL sources \cite{ArtemFOA}.
Opening the field of attoscience to the xFEL machine, one could even imagine an attosecond resolved extension to the two-dimensional resonant Auger electron spectrosopies of Ref.~\cite{Piancastelli2013}.

\begin{wrapfigure}[20]{l}{.5\linewidth}
\centerline{\includegraphics[width=.75\linewidth]{1-s2.0-S0009261407015436-main.terrainmap.eps}}
\caption{\label{fig::cederbaum}Hole migration in PENNA molecule following photoionization in the ground neutral molecular configuration (top) versus the C$_2$-C$_2$ 20pm stretched configuration reproduced from Ref.~\cite{Cederbaum2008}.}
\end{wrapfigure}

One of the biggest challenges for the traditionally laser-based attosecond science community is the difficulty in producing significant pulse energy in the 200 eV -- 2 keV regime using high harmonic generation (HHG)\cite{Chen2014,Schmidt2016}.
These sources are encroaching on this range \cite{Biegert2014,Zenghu2017}, however the intensities and repetition rates available at FEL sources like the LCLS-II (Table \ref{lcls2specs}) are driving the community to look toward all x-ray pump-probe. 
There have been numerous schemes proposed for developing the attosecond capability of x-ray FEL facilities \cite{Ding2009,Xiang2009} with a particular push funded directly by the Office of Basic Energy Science aimed at LCLS-II implementation \cite{Marinelli2016,xLEAP}.


Achieving the full capability of attosecond x-ray laser science will require the diagnosis and control of the x-ray spectral phase.
The development of temporally shaped x-ray FEL pulses would not only facilitate attosecond pulse generation but also a number of multi-pulse non-linear techniques.
A continued progress on this front \cite{Lutman13_twocolor,Marinelli13_twocolor,Allaria2014,Marinelli2015,Prince2016,Lutman2016,Marinelli2016} will require full spectral phase, amplitude and polarization characterization as recently demonstrated in Ref.~\cite{Nick2018}.
We therefore propose a single-shot diagnostic that reports the full temporal intensity, wavelength, and polarization distributions with $\sim$150 attoseconds resolution at the highest repetition rates, limited only by the optical laser repetition rate that is used for the streaking drive laser.

We restrict ourselves to two fundamental objectives:
\begin{enumerate}
\item \label{obj::detector} We will perform the necessary reasearch and development required to eventually deliver an optimized angular array of electron Time-of-Flight spectrometers.
\item \label{obj::analysis} We will co-develop the requisite algorithms and machine learning compliments together with the detector electronics such in odred to optimally match the real-time analysis routines with detector electronics and computing hardware that push the tenchnology envelope.
\end{enumerate}

